\import{set.tex}

\begin{proposition}\label{times_subseteq_left}
    Suppose $A\subseteq C$. Then $A\times B\subseteq C\times B$.
\end{proposition}
\begin{proof}
    It suffices to show that for all $w\in A\times B$ we have $w\in C\times B$.
\end{proof}

\begin{proposition}\label{times_subseteq_right}
    Suppose $B\subseteq D$. Then $A\times B\subseteq A\times D$.
\end{proposition}
\begin{proof}
    It suffices to show that for all $w\in A\times B$ we have $w\in A\times D$.
\end{proof}

\begin{proposition}\label{inter_times_intro}
    Suppose $w\in(A\inter B)\times (C\inter D)$.
    Then $w\in(A\times C)\inter (B\times D)$.
\end{proposition}
\begin{proof}
    Take $a,c$ such that $w = (a, c)$
        by \cref{times_elem_is_tuple}.
    Then $a\in A, B$ and $c\in C,D$
        by \cref{times_tuple_elim,inter}.
    Thus $w\in (A\times C), (B\times D)$.
\end{proof}

\begin{proposition}\label{inter_times_elim}
    Suppose $w\in(A\times C)\inter (B\times D)$.
    Then $w\in(A\inter B)\times (C\inter D)$.
\end{proposition}
\begin{proof}
    $w\in A\times C$.
    Take $a, c$ such that $w = (a, c)$.
    $a\in A, B$ by \cref{inter,times_tuple_elim}.
    $c\in C, D$ by \cref{inter,times_tuple_elim}.
    Thus $(a,c) \in (A\inter B)\times (C\inter D)$ by \cref{times,inter_intro}.
\end{proof}

\begin{proposition}\label{inter_times}
    $(A\inter B)\times (C\inter D) = (A\times C)\inter (B\times D)$.
\end{proposition}
\begin{proof}
    Follows by set extensionality.
\end{proof}

\begin{proposition}\label{inter_times_right}
    $(X\inter Y)\times Z = (X\times Z)\inter (Y\times Z)$.
\end{proposition}
\begin{proof}
    Follows by set extensionality.
\end{proof}

\begin{proposition}\label{inter_times_left}
    $X\times (Y\inter Z) = (X\times Y)\inter (X\times Z)$.
\end{proposition}
\begin{proof}
    Follows by set extensionality.
\end{proof}

\begin{proposition}\label{union_times_intro}
    Suppose $w\in(A\union B)\times (C\union D)$.
    Then $w\in(A\times C)\union (B\times D)\union (A\times D)\union (B\times C)$.
\end{proposition}
\begin{proof}
    Take $a,c$ such that $w = (a, c)$.
    $a\in A$ or $a\in B$ by \cref{union_iff,times_tuple_elim}.
    $c\in C$ or $c\in D$ by \cref{union_iff,times_tuple_elim}.
    Thus $(a, c)\in (A\times C)$ or $(a, c)\in (B\times D)$ or $(a, c)\in (A\times D)$ or $(a, c)\in (B\times C)$.
    Thus $(a, c)\in (A\times C)\union (B\times D)\union (A\times D)\union (B\times C)$.
\end{proof}

\begin{proposition}\label{union_times_elim}
    Suppose $w\in(A\times C)\union (B\times D)\union (A\times D)\union (B\times C)$.
    Then $w\in(A\union B)\times (C\union D)$.
\end{proposition}
\begin{proof}
    \begin{byCase}
        \caseOf{$w\in(A\times C)$.}
            Take $a, c$ such that $w = (a, c) \land a\in A\land c\in C$ by \cref{times}.
            Then $a\in A\union B$ and $c\in C\union D$.
            Follows by \cref{times_tuple_intro}.
        \caseOf{$w\in(B\times D)$.}
            Take $b, d$ such that $w = (b, d) \land b\in B\land d\in D$ by \cref{times}.
            Then $b\in A\union B$ and $d\in C\union D$.
            Follows by \cref{times_tuple_intro}.
        \caseOf{$w\in(A\times D)$.}
            Take $a, d$ such that $w = (a, d) \land a\in A\land d\in D$ by \cref{times}.
            Then $a\in A\union B$ and $d\in C\union D$.
            Follows by \cref{times_tuple_intro}.
        \caseOf{$w\in(B\times C)$.}
            Take $b, c$ such that $w = (b, c) \land b\in B\land c\in C$ by \cref{times}.
            Then $b\in A\union B$ and $c\in C\union D$.
            Follows by \cref{times_tuple_intro}.
    \end{byCase}
\end{proof}

\begin{proposition}\label{union_times}
    $(A\union B)\times (C\union D) = (A\times C)\union (B\times D)\union (A\times D)\union (B\times C)$.
\end{proposition}
\begin{proof}
    Follows by set extensionality.
\end{proof}

\begin{proposition}\label{union_times_left}
    $(X\union Y)\times Z = (X\times Z)\union (Y\times Z)$.
\end{proposition}
\begin{proof}
    Follows by set extensionality.
\end{proof}

\begin{proposition}\label{union_times_right}
    $X\times (Y\union Z) = (X\times Y)\union (X\times Z)$.
\end{proposition}
\begin{proof}
    Follows by set extensionality.
\end{proof}
