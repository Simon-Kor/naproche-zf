\import{set.tex}
\import{set/powerset.tex}

\section{Filters}

\subsection{Definition and basic properties of filters}

\begin{abbreviation}\label{upwardclosed}
    $F$ is upward-closed in $S$ iff
    for all $A, B$ such that $A\subseteq B\subseteq S$ and $A\in F$ we have $B\in F$.
\end{abbreviation}

\begin{definition}\label{filter}
    $F$ is a filter on $S$ iff
    $F$ is a family of subsets of $S$
    and $S\in F$
    and $\emptyset\notin F$
    and $F$ is closed under binary intersections
    and $F$ is upward-closed in $S$.
\end{definition}

\begin{proposition}\label{filter_ext_complement}
    Let $F, G$ be filters on $S$.
    Suppose for all $A\subseteq S$ we have $S\setminus A\in F$ iff $S\setminus A\in G$.
    Then $F = G$.
\end{proposition}
\begin{proof}
    Follows by set extensionality.
\end{proof}

\begin{proposition}\label{filter_inter_in_iff}
    Let $F$ be a filter on $S$.
    Suppose $A, B\subseteq S$.
    Then $A\inter B\in F$ iff $A, B\in F$.
\end{proposition}
\begin{proof}
    We have $A\inter B\subseteq A, B$.
    Follows by \cref{filter}.
\end{proof}

\begin{proposition}\label{filter_setminus_in}
    Let $F$ be a filter on $S$.
    Suppose $A\in F$.
    Suppose $B\subseteq S$ and $S\setminus B\in F$.
    Then $A\setminus B\in F$.
\end{proposition}
\begin{proof}
    We have $A\subseteq S$.
    Thus $A\setminus B = A\inter (S\setminus B)$ by \cref{setminus_eq_inter_complement}.
    Now $S\setminus B\subseteq S$.
    Follows by \cref{filter_inter_in_iff}.
\end{proof}

\begin{proposition}\label{filter_in_iff_exists_subset}
    Let $F$ be a filter on $S$.
    Suppose $B\subseteq S$.
    Then $B\in F$ iff there exists $A\subseteq B$ such that $A\in F$.
\end{proposition}


\subsection{Principal filters over a set}

\begin{definition}\label{principalfilter}
    $\principalfilter{S}{A} = \{X\in\pow{S}\mid A\subseteq X\}$.
\end{definition}

\begin{proposition}\label{principalfilter_iff}
    Suppose $A, B\subseteq S$.
    Then $B\in\principalfilter{S}{A}$ iff $A\subseteq B$.
\end{proposition}

\begin{proposition}\label{principalfilter_bottom}
    Suppose $A\subseteq S$.
    Then $A\in\principalfilter{S}{A}$.
\end{proposition}

\begin{proposition}\label{principalfilter_top}
    Suppose $A\subseteq S$.
    Then $S\in\principalfilter{S}{A}$.
\end{proposition}

\begin{proposition}\label{principalfilter_is_filter}
    Suppose $A\subseteq S$.
    Suppose $A$ is inhabited.
    Then $\principalfilter{S}{A}$ is a filter on $S$.
\end{proposition}
\begin{proof}
    $S$ is inhabited. % since $A$ is inhabited and $A\subseteq S$.
    $\principalfilter{S}{A}$ is a family of subsets of $S$.
    $S\in \principalfilter{S}{A}$.
    $\emptyset\notin \principalfilter{S}{A}$.
    $\principalfilter{S}{A}$ is closed under binary intersections.
    $\principalfilter{S}{A}$ is upward-closed in $S$.
    Follows by \cref{filter}.
\end{proof}

\begin{proposition}\label{principalfilter_elem_generator}
    Suppose $A\subseteq S$.
    $A\in\principalfilter{S}{A}$.
\end{proposition}
\begin{proof}
    $A\in\pow{S}$.
\end{proof}

\begin{proposition}\label{principalfilter_notelem_implies_notsupseteq}
    Let $X\in\pow{S}$.
    Suppose $X\notin\principalfilter{S}{A}$.
    Then $A\not\subseteq X$.
\end{proposition}

\begin{definition}\label{maximalfilter}
    $F$ is a maximal filter on $S$ iff
    $F$ is a filter on $S$ and there exists no filter $F'$ on $S$ such that $F\subset F'$.
\end{definition}

\begin{proposition}\label{principalfilter_singleton_is_filter}
    Suppose $a\in S$.
    Then $\principalfilter{S}{\{a\}}$ is a filter on $S$.
\end{proposition}
\begin{proof}
    $\{a\}\subseteq S$.
    $\{a\}$ is inhabited.
    Follows by \cref{principalfilter_is_filter}.
\end{proof}

\begin{proposition}\label{principalfilter_singleton_is_maximal_filter}
    Suppose $a\in S$.
    Then $\principalfilter{S}{\{a\}}$ is a maximal filter on $S$.
\end{proposition}
\begin{proof}
    $\{a\}\subseteq S$.
    $\{a\}$ is inhabited.
    Thus $\principalfilter{S}{\{a\}}$ is a filter on $S$ by \cref{principalfilter_is_filter}.
    It suffices to show that there exists no filter $F'$ on $S$ such that $\principalfilter{S}{\{a\}}\subset F'$.
    Suppose not.
    Take a filter $F'$ on $S$ such that $\principalfilter{S}{\{a\}}\subset F'$.
    Take $X\in F'$ such that $X\notin \principalfilter{S}{\{a\}}$.
    $X\in\pow{S}$.
    Thus $\{a\}\not\subseteq X$ by \cref{principalfilter_notelem_implies_notsupseteq}.
    Thus $a\notin X$.
    $\{a\}\in F'$
        % TODO clean
        by \cref{principalfilter,elem_subseteq,union_intro_left,subset,union_absorb_subseteq_left,subseteq_refl,powerset_intro}.
    Thus $\emptyset = X\inter \{a\}$.
    Hence $\emptyset\in F'$ by \cref{filter}.
    Follows by \hyperref[filter]{contradiction to the definition of a filter}.
\end{proof}
