\import{set.tex}
\import{set/cons.tex}
\import{set/powerset.tex}
\import{set/fixpoint.tex}
\import{set/product.tex}
\import{topology/topological-space.tex}
\import{topology/separation.tex}
\import{topology/continuous.tex}
\import{topology/basis.tex}
\import{numbers.tex}
\import{function.tex}


\section{Topology Reals}

\begin{definition}\label{topological_basis_reals_eps_ball}
    $\topoBasisReals = \{ \epsBall{x}{\epsilon} \mid x \in \reals, \epsilon \in \realsplus\}$.
\end{definition}

\begin{axiom}\label{reals_carrier_reals}
    $\carrier[\reals] = \reals$.
\end{axiom}

\begin{lemma}\label{intervals_are_connected_in_reals}
    Suppose $a,b \in \reals$.
    Then for all $c \in \reals$ such that $a < c < b$ we have $c \in \intervalopen{a}{b}$.
\end{lemma}

\begin{lemma}\label{epsball_are_subset_reals_elem}
    Suppose $x \in \reals$.
    Suppose $\epsilon \in \realsplus$.
    Then for all $y \in \epsBall{x}{\epsilon}$ we have $y \in \reals$.
\end{lemma}

\begin{lemma}\label{intervalopen_iff}
    Suppose $a,b,c \in \reals$.
    Suppose $a < b$.
    $c \in \intervalopen{a}{b}$ iff $a < c < b$.
\end{lemma}

\begin{lemma}\label{epsball_are_subseteq_reals_set}
    Suppose $x \in \reals$.
    Suppose $\epsilon \in \realsplus$.
    Then $\epsBall{x}{\epsilon} \subseteq \reals$.
\end{lemma}

\begin{lemma}\label{epsball_are_subset_reals_set}
    Suppose $x \in \reals$.
    Suppose $\epsilon \in \realsplus$.
    Then $\epsBall{x}{\epsilon} \subset \reals$.
\end{lemma}

\begin{lemma}\label{reals_order_minus_positiv}
    Suppose $x,y \in \reals$.
    Suppose $\zero < y$.
    $x - y < x$.
\end{lemma}

\begin{lemma}\label{realsplus_bigger_zero}
    For all $x \in \realsplus$ we have $\zero < x$.
\end{lemma}

\begin{lemma}\label{realsplus_in_reals}
    For all $x \in \realsplus$ we have $x \in \reals$.
\end{lemma}

\begin{lemma}\label{epsball_are_inhabited}
    Suppose $x \in \reals$.
    Suppose $\epsilon \in \realsplus$.
    Then $\epsBall{x}{\epsilon}$ is inhabited.
\end{lemma}
\begin{proof}
    $x < x + \epsilon$ by \cref{reals_order_behavior_with_addition,realsplus,reals_axiom_zero_in_reals,reals_axiom_kommu,reals_axiom_zero}.
    $x - \epsilon < x$.
    $x \in \epsBall{x}{\epsilon}$.
\end{proof}

\begin{lemma}\label{reals_elem_inbetween}
    For all $a,b \in \reals$ such that $a < b$ we have there exists $c \in \reals$ such that $a < c < b$.
\end{lemma}

\begin{lemma}\label{epsball_equal_openinterval}
    Suppose $x \in \reals$.
    Suppose $\epsilon \in \realsplus$.
    Then $\epsBall{x}{\epsilon} = \intervalopen{x - \epsilon}{x + \epsilon}$.
\end{lemma}

\begin{lemma}\label{minus_behavior1}
    For all $x \in \reals$ we have $x - x = \zero$.
\end{lemma}

\begin{lemma}\label{minus_behavior2}
    For all $x \in \reals$ we have $x + \neg{x} = \zero$.
\end{lemma}

\begin{lemma}\label{minus_behavior3}
    For all $x \in \reals$ we have $\neg{x} = \zero - x$.
\end{lemma}

\begin{lemma}\label{reals_order_is_addition_with_positiv_number}
    For all $x,y \in \reals$ such that $x < y$ we have there exists $z \in \realsplus$ such that $x + z = y$.
\end{lemma}
\begin{proof}
    Omitted.
\end{proof}




\begin{lemma}\label{reals_order_plus_minus}
    Suppose $a,b \in \reals$.
    Suppose $\zero < b$.
    Then $(a-b) < (a+b)$.
\end{lemma}
\begin{proof}
    We show that $a < (a+b)$.
    \begin{subproof}
        Trivial.
    \end{subproof}
    We show that $(a-b) < a$.
    \begin{subproof}
        Trivial.
    \end{subproof}
\end{proof}

\begin{lemma}\label{epsball_are_connected_in_reals}
    Suppose $x \in \reals$.
    Suppose $\epsilon \in \realsplus$.
    Then for all $c \in \reals$ such that $(x - \epsilon) < c < (x + \epsilon)$ we have $c \in \epsBall{x}{\epsilon}$.
\end{lemma}
\begin{proof}
    $x - \epsilon \in \reals$.
    $x + \epsilon \in \reals$.


    It suffices to show that for all $c$ such that $c \in \reals \land (x - \epsilon) \rless c \rless (x + \epsilon)$ we have $c \in \epsBall{x}{\epsilon}$.
    Fix $c$ such that $(c \in \reals) \land (x - \epsilon) \rless c \rless (x + \epsilon)$.
    $(x - \epsilon) < c < (x + \epsilon)$.
\end{proof}

\begin{theorem}\label{topological_basis_reals_is_prebasis}
    $\topoBasisReals$ is a topological prebasis for $\reals$.
\end{theorem}
\begin{proof}
    We show that $\unions{\topoBasisReals} \subseteq \reals$.
    \begin{subproof}
        It suffices to show that for all $x \in \unions{\topoBasisReals}$ we have $x \in \reals$.
        Fix $x \in \unions{\topoBasisReals}$.
        \begin{byCase}
            \caseOf{$x = \emptyset$.}
                Trivial.
            \caseOf{$x \neq \emptyset$.}
                %There exists $U \in \topoBasisReals$ such that $x \in U$.
                Take $U \in \topoBasisReals$ such that $x \in U$.
                Follows by \cref{epsball_are_subseteq_reals_set,topological_basis_reals_eps_ball,epsilon_ball,minus,subseteq}.
        \end{byCase}
    \end{subproof}
    We show that $\reals \subseteq \unions{\topoBasisReals}$.
    \begin{subproof}
        It suffices to show that for all $x \in \reals$ we have $x \in \unions{\topoBasisReals}$.
        Fix $x \in \reals$.
        $\epsBall{x}{1} \in \topoBasisReals$.
        Therefore $x \in \unions{\topoBasisReals}$ by \cref{one_in_reals,reals_one_bigger_zero,unions_intro,realsplus,plus_one_order,reals_order_minus_positiv,epsball_are_connected_in_reals}.
    \end{subproof}
\end{proof}

%\begin{lemma}\label{intervl_intersection_is_interval}
%    Suppose $a,b,a',b' \in \reals$.
%    Suppose there exist $x \in \reals$ such that $x \in \intervalopen{a}{b} \inter \intervalopen{a'}{b'}$.
%    Then there exists $q,p \in \reals$ such that $q < p$ and $\intervalopen{q}{p} \subseteq \intervalopen{a}{b} \inter \intervalopen{a'}{b'}$.
%\end{lemma}
%

\begin{lemma}\label{reals_order_total}
    For all $x,y \in \reals$ we have either $x < y$ or $x \geq y$. 
\end{lemma}
\begin{proof}
    It suffices to show that for all $x \in \reals$ we have for all $y \in \reals$ we have either $x < y$ or $x \geq y$. 
    Fix $x \in \reals$.
    Fix $y \in \reals$.
    Omitted.
\end{proof}

\begin{lemma}\label{topo_basis_reals_eps_iff}
    $X \in \topoBasisReals$ iff there exists $x_0, \delta$ such that $x_0 \in \reals$ and $\delta \in \realsplus$ and $\epsBall{x_0}{\delta} = X$.
\end{lemma}

\begin{lemma}\label{topo_basis_reals_intro}
For all $x,\delta$ such that $x \in \reals \land \delta \in \realsplus$ we have $\epsBall{x}{\delta} \in \topoBasisReals$.
\end{lemma}

\begin{lemma}\label{realsplus_in_reals_plus}
    For all $x,y$ such that $x \in \reals$ and $y \in \realsplus$ we have $x + y \in \reals$.
\end{lemma}

\begin{lemma}\label{realspuls_in_reals_minus}
    For all $x,y$ such that $x \in \reals$ and $y \in \realsplus$ we have $x - y \in \reals$.
\end{lemma}

\begin{lemma}\label{eps_ball_implies_open_interval}
    Suppose $x \in \reals$.
    Suppose $\epsilon \in \realsplus$. 
    Then there exists $a,b \in \reals$ such that $a < b$ and $\intervalopen{a}{b} = \epsBall{x}{\epsilon}$.
\end{lemma}

\begin{lemma}\label{one_in_realsplus}
    $1 \in \realsplus$.
\end{lemma}

\begin{lemma}\label{reals_existence_addition_reverse}
    For all $\delta \in \reals$ there exists $\epsilon \in \reals$ such that $\epsilon + \epsilon = \delta$.
\end{lemma}
\begin{proof}
    Fix $\delta \in \reals$.
    Follows by \cref{one_in_realsplus,reals_disstro2,reals_axiom_disstro1,reals_rmul,suc_eq_plus_one,reals_axiom_mul_invers,suc,suc_neq_emptyset,realsplus_in_reals_plus,naturals_addition_axiom_2,naturals_1_kommu,reals_axiom_zero,naturals_inductive_set,one_is_suc_zero,realsplus,reals_one_bigger_zero,one_in_reals,reals_axiom_one,minus_in_reals}.
\end{proof}

\begin{lemma}\label{reals_addition_minus_behavior1}
    For all $a,b,c \in \reals$ such that $a = b + c$ we have $b = a - c$.
\end{lemma}
\begin{proof}
    It suffices to show that for all $a \in \reals$ for all $b \in \reals$ for all $c \in \reals$ if $a = b + c$ then $b = a - c$.
    Fix $a \in \reals$.
    Fix $b \in \reals$.
    Fix $c \in \reals$.
    Suppose $a = b + c$.
    Then $a + \neg{c} = b + c + \neg{c}$.
    Therefore $a - c = b + c + \neg{c}$.
    $a - c = (b + c) - c$.
    $(b + c) - c = (b + c) + \neg{c}$.
    $(b + c) + \neg{c} = b + (c + \neg{c})$.
    $b + (c + \neg{c}) = b + (\zero)$.
    $a - c = b$.
\end{proof}

\begin{lemma}\label{reals_addition_minus_behavior2}
    For all $a,b,c \in \reals$ such that $a = b - c$ we have $b = c + a$.
\end{lemma}

\begin{lemma}\label{open_interval_eq_eps_ball}
    Suppose $a,b \in \reals$.
    Suppose $a < b$.
    Then there exist $x,\epsilon$ such that $x \in \reals$ and $\epsilon \in \realsplus$ and $\intervalopen{a}{b} = \epsBall{x}{\epsilon}$.
\end{lemma}
\begin{proof}
    Let $\delta = (b-a)$.
    $\delta$ is positiv by \cref{minus_in_reals,minus_behavior3,reals_axiom_zero_in_reals,reals_order_behavior_with_addition,minus_behavior1,minus}.
    There exists $\epsilon \in \reals$ such that $\epsilon + \epsilon = \delta$.
    Let $x = a + \epsilon$.
    $a + \delta = b$.
    $a + \epsilon + \epsilon = b$. 
    $x + \epsilon = b$.
    $\epsilon \in \realsplus$ by \cref{reals_order_behavior_with_addition,reals_axiom_kommu,reals_axiom_zero,reals_order_is_transitive,reals_add,minus_behavior1,minus_behavior3,minus,reals_order_total,reals_axiom_zero_in_reals,realsplus}.
    $a = x - \epsilon$.
    $b = x + \epsilon$.
    We show that $\intervalopen{a}{b} \subseteq \epsBall{x}{\epsilon}$.
    \begin{subproof}
        It suffices to show that for all $y \in \intervalopen{a}{b}$ we have $y \in \epsBall{x}{\epsilon}$.
        Fix $y \in \intervalopen{a}{b}$.
    \end{subproof}
    We show that $\epsBall{x}{\epsilon} \subseteq \intervalopen{a}{b}$.
    \begin{subproof}
        It suffices to show that for all $y \in \epsBall{x}{\epsilon}$ we have $y \in \intervalopen{a}{b}$ by \cref{subseteq}.
        Fix $y \in \epsBall{x}{\epsilon}$.
    \end{subproof}

\end{proof}

\begin{lemma}\label{intersection_openinterval_inclusion_of_border}
    Suppose $a,b,x,y \in \reals$.
    Suppose $a < b$.
    Suppose $x < y$.
    Suppose $a \leq x < y \leq b$.
    Then $\intervalopen{a}{b} \inter \intervalopen{x}{y} = \intervalopen{x}{y}$.
\end{lemma}
\begin{proof}
    Omitted.
\end{proof}

\begin{lemma}\label{intersection_openinterval_lower_border_eq}
    Suppose $a,b,x,y \in \reals$.
    Suppose $a < b$.
    Suppose $x < y$.
    Suppose $a = x$ and $b \leq y$.
    Then $\intervalopen{a}{b} \inter \intervalopen{x}{y} = \intervalopen{a}{b}$.
\end{lemma}
\begin{proof}
    Omitted.
\end{proof}

\begin{lemma}\label{intersection_openinterval_upper_border_eq}
    Suppose $a,b,x,y \in \reals$.
    Suppose $a < b$.
    Suppose $x < y$.
    Suppose $a \leq x$ and $b = y$.
    Then $\intervalopen{a}{b} \inter \intervalopen{x}{y} = \intervalopen{x}{y}$.
\end{lemma}
\begin{proof}
    Omitted.
\end{proof}

\begin{lemma}\label{intersection_openinterval_none_border_eq}
    Suppose $a,b,x,y \in \reals$.
    Suppose $a < b$.
    Suppose $x < y$.
    If $a \leq x < b \leq y$ then $\intervalopen{a}{b} \inter \intervalopen{x}{y} = \intervalopen{x}{b}$.
\end{lemma}
\begin{proof}
    Omitted.
\end{proof}

\begin{lemma}\label{reals_order_total2}
    For all $a,b \in \reals$ we have $a < b \lor a > b \lor a = b$.
\end{lemma}


\begin{theorem}\label{topological_basis_reals_is_basis}
    $\topoBasisReals$ is a topological basis for $\reals$.
\end{theorem}
\begin{proof}
    $\topoBasisReals$ is a topological prebasis for $\reals$ by \cref{topological_basis_reals_is_prebasis}.
    Let $B = \topoBasisReals$.
    It suffices to show that for all $U \in B$ we have for all $V \in B$ we have for all $x$ such that $x \in U, V$ there exists $W\in B$ such that $x\in W\subseteq U, V$.
    Fix $U \in B$.
    Fix $V \in B$.
    It suffices to show that for all $x \in U \inter V$ there exists $W\in B$ such that $x\in W\subseteq U, V$.
    Fix $x \in U \inter V$.
    \begin{byCase}
        \caseOf{$U \inter V = \emptyset$.}
            Trivial.
        \caseOf{$U \inter V \neq \emptyset$.}
            Then $U \inter V$ is inhabited.
            $x \in \reals$ by \cref{inter_lower_left,subseteq,topological_prebasis_iff_covering_family,omega_is_an_ordinal,naturals_subseteq_reals,subset_transitive,suc_subseteq_elim,ordinal_suc_subseteq}.
            There exists $x_1, \alpha$ such that $x_1 \in \reals$ and $\alpha \in \realsplus$ and $\epsBall{x_1}{\alpha} = U$.
            There exists $x_2, \beta$ such that $x_2 \in \reals$ and $\beta \in \realsplus$ and $\epsBall{x_2}{\beta} = V$.
            Then $ (x_1 - \alpha) < x < (x_1 + \alpha)$.
            Then $ (x_2 - \beta) < x < (x_2 + \beta)$.
            Let $a = (x_1 - \alpha)$.
            Let $b = (x_1 + \alpha)$.
            Let $c = (x_2 - \beta)$.
            Let $d = (x_2 + \beta)$.
            We have $a < b$ and $a < x$ and $x < b$.
            We have $c < d$ and $c < x$ and $x < d$.
            We have $a \in \reals$.
            We have $b \in \reals$.
            We have $c \in \reals$.
            We have $d \in \reals$.
            We show that there exist $a',b'\in \reals$ such that $\intervalopen{a}{b} \inter \intervalopen{c}{d} = \intervalopen{a'}{b'}$.
            \begin{subproof} 
                \begin{byCase}
                    \caseOf{$a < c$.}
                        \begin{byCase}
                            \caseOf{$b < d$.}
                                Follows by \cref{intersection_openinterval_inclusion_of_border,intersection_openinterval_lower_border_eq,intersection_openinterval_none_border_eq,intersection_openinterval_upper_border_eq,reals_order_is_transitive,realsplus_in_reals_plus,realspuls_in_reals_minus}.
                            \caseOf{$b = d$.}
                                Follows by \cref{intersection_openinterval_inclusion_of_border,intersection_openinterval_lower_border_eq,intersection_openinterval_none_border_eq,intersection_openinterval_upper_border_eq,reals_order_is_transitive,realsplus_in_reals_plus,realspuls_in_reals_minus}.
                            \caseOf{$b > d$.}
                                Follows by \cref{intersection_openinterval_inclusion_of_border,intersection_openinterval_lower_border_eq,intersection_openinterval_none_border_eq,intersection_openinterval_upper_border_eq,reals_order_is_transitive,realsplus_in_reals_plus,realspuls_in_reals_minus}.
                        \end{byCase}
                    \caseOf{$a = c$.}
                        \begin{byCase}
                            \caseOf{$b < d$.}
                                Follows by \cref{intersection_openinterval_inclusion_of_border,intersection_openinterval_lower_border_eq,intersection_openinterval_none_border_eq,intersection_openinterval_upper_border_eq,reals_order_is_transitive,realsplus_in_reals_plus,realspuls_in_reals_minus}.
                            \caseOf{$b = d$.}
                                Follows by \cref{intersection_openinterval_inclusion_of_border,intersection_openinterval_lower_border_eq,intersection_openinterval_none_border_eq,intersection_openinterval_upper_border_eq,reals_order_is_transitive,realsplus_in_reals_plus,realspuls_in_reals_minus}.
                            \caseOf{$b > d$.}
                                Follows by \cref{intersection_openinterval_inclusion_of_border,intersection_openinterval_lower_border_eq,intersection_openinterval_none_border_eq,intersection_openinterval_upper_border_eq,reals_order_is_transitive,realsplus_in_reals_plus,realspuls_in_reals_minus}.
                        \end{byCase}
                    \caseOf{$a > c$.}
                        \begin{byCase}
                            \caseOf{$b < d$.}
                                Follows by \cref{intersection_openinterval_inclusion_of_border,intersection_openinterval_lower_border_eq,intersection_openinterval_none_border_eq,intersection_openinterval_upper_border_eq,reals_order_is_transitive,realsplus_in_reals_plus,realspuls_in_reals_minus,reals_add,minus_in_reals,realsplus,inter_comm,epsilon_ball}.
                            \caseOf{$b = d$.}
                                Follows by \cref{intersection_openinterval_inclusion_of_border,intersection_openinterval_lower_border_eq,intersection_openinterval_none_border_eq,intersection_openinterval_upper_border_eq,reals_order_is_transitive,realsplus_in_reals_plus,realspuls_in_reals_minus,reals_add,minus_in_reals,realsplus,inter_comm,epsilon_ball}.
                            \caseOf{$b > d$.}
                                Follows by \cref{intersection_openinterval_inclusion_of_border,intersection_openinterval_lower_border_eq,intersection_openinterval_none_border_eq,intersection_openinterval_upper_border_eq,reals_order_is_transitive,realsplus_in_reals_plus,realspuls_in_reals_minus,reals_add,minus_in_reals,realsplus,inter_comm,epsilon_ball}.
                        \end{byCase}
                \end{byCase}
            \end{subproof}

            Take $a',b'\in \reals$ such that $\intervalopen{a}{b} \inter \intervalopen{c}{d} = \intervalopen{a'}{b'}$.
            We have $a',b' \in \reals$ by assumption.
            We have $a' < b'$ by \cref{id_img,epsilon_ball,minus,intervalopen,reals_order_is_transitive}.
            Then there exist $x',\epsilon'$ such that $x' \in \reals$ and $\epsilon' \in \realsplus$ and $\intervalopen{a'}{b'} = \epsBall{x'}{\epsilon'}$.
            Then $x \in \epsBall{x'}{\epsilon'}$ by \cref{epsilon_ball}.

            Follows by \cref{inter_lower_left,inter_lower_right,epsilon_ball,topological_basis_reals_eps_ball}.
            %Then $(x_1 - \alpha) < (x_2 + \beta)$.

            %Therefore $U \inter V = \intervalopen{}{(x_2 + \beta)}$.

            %We show that if there exists $\delta \in \realsplus$ such that $\epsBall{x}{\delta} \subseteq U \inter V$ then there exists $W\in B$ such that $x\in W\subseteq U, V$.
            %\begin{subproof}
            %    Suppose there exists $\delta \in \realsplus$ such that $\epsBall{x}{\delta} \subseteq U \inter V$.
            %    $x \in \epsBall{x}{\delta}$.
            %    $\epsBall{x}{\delta} \subseteq U$.
            %    $\epsBall{x}{\delta} \subseteq V$.
            %    $\epsBall{x}{\delta} \in B$.
            %\end{subproof}
            %It suffices to show that there exists $\delta \in \realsplus$ such that $\epsBall{x}{\delta} \subseteq U \inter V$.


            %It suffices to show that there exists $x_0, \delta$ such that $x_0 \in \reals$ and $\delta \in \realsplus$ and $\epsBall{x_0}{\delta} \subseteq u \inter V$.
            %There exists $x_1, \alpha$ such that $x_1 \in \reals$ and $\alpha \in \realsplus$ and $\epsBall{x_1}{\alpha} = U$.
            %There exists $x_2, \beta$ such that $x_2 \in \reals$ and $\beta \in \realsplus$ and $\epsBall{x_2}{\beta} = V$.
            %Then $ (x_1 - \alpha) < x < (x_1 + \alpha)$.
            %Then $ (x_2 - \beta) < x < (x_2 + \beta)$.
            %\begin{byCase}
            %    \caseOf{$x_1 = x_2$.}
            %        Take $\gamma \in \realsplus$ such that either $\gamma = \alpha \land \gamma \leq \beta$ or $\gamma \leq \alpha \land \gamma = \beta$.
            %    \caseOf{$x_1 < x_2$.}
            %    \caseOf{$x_1 > x_2$.}
            %\end{byCase}
            %%Take $m$ such that $m \in \min{\{(x_1 + \alpha), (x_2 + \beta)\}}$.
            %Take $n$ such that $n \in \max{\{(x_1 - \alpha), (x_2 - \beta)\}}$.
            %Then $m < x < n$.
            %We show that there exists $x_1 \in \reals$ such that $x_1 \in U \inter V$ and $x_1 < x$.
            %\begin{subproof}
            %    Suppose not.
            %    Then For all $y \in U \inter V$ we have $x \leq y$.
            %\end{subproof}
            %We show that there exists $x_2 \in \reals$ such that $x_2 \in U \inter V$ and $x_2 > x$.
            %\begin{subproof}
            %    Trivial.
            %\end{subproof}
    \end{byCase}
\end{proof}

\begin{axiom}\label{topological_space_reals}
    $\opens[\reals] = \genOpens{\topoBasisReals}{\reals}$.
\end{axiom}

\begin{theorem}\label{reals_is_topological_space}
    $\reals$ is a topological space.
\end{theorem}
\begin{proof}
    $\topoBasisReals$ is a topological basis for $\reals$.
    Let $B = \topoBasisReals$.
    We show that $\opens[\reals]$ is a family of subsets of $\carrier[\reals]$.
    \begin{subproof}
        It suffices to show that for all $A \in \opens[\reals]$ we have $A \subseteq \reals$.
        Fix $A \in \opens[\reals]$.
        Follows by \cref{powerset_elim,topological_space_reals,genopens}.
    \end{subproof}
    We show that $\reals \in\opens[\reals]$.
    \begin{subproof}
        $B$ covers $\reals$ by \cref{topological_prebasis_iff_covering_family,topological_basis}.
        $\unions{B} \in \genOpens{B}{\reals}$.
        $\reals \subseteq \unions{B}$.
    \end{subproof}
    We show that for all $A, B\in \opens[\reals]$ we have $A\inter B\in\opens[\reals]$.
    \begin{subproof}
        Follows by \cref{topological_space_reals,inters_in_genopens}.
    \end{subproof}
    We show that for all $F\subseteq \opens[\reals]$ we have $\unions{F}\in\opens[\reals]$.
    \begin{subproof}
        Follows by \cref{topological_space_reals,union_in_genopens}.
    \end{subproof}
    $\carrier[\reals] = \reals$.
    Follows by \cref{topological_space}.
\end{proof}

\begin{proposition}\label{open_interval_is_open}
    Suppose $a,b \in \reals$.
    Then $\intervalopen{a}{b} \in \opens[\reals]$.
\end{proposition}
\begin{proof}
    If $a > b$ then $\intervalopen{a}{b} = \emptyset$.
    If $a = b$ then $\intervalopen{a}{b} = \emptyset$.
    It suffices to show that if $a < b$ then $\intervalopen{a}{b} \in \opens[\reals]$.
    Suppose $a \rless b$.
    Take $x, \epsilon$ such that $x \in \reals$ and $\epsilon \in \realsplus$ and $\intervalopen{a}{b} = \epsBall{x}{\epsilon}$.
    It suffices to show that $\epsBall{x}{\epsilon} \in \opens[\reals]$.
    $\topoBasisReals$ is a topological basis for $\reals$.
    $\epsBall{x}{\epsilon} \in \topoBasisReals$.
    $\topoBasisReals \subseteq \opens[\reals]$ by \cref{basis_is_in_genopens,topological_space_reals,topological_basis_reals_is_basis}.
\end{proof}

\begin{lemma}\label{reals_minus_to_realsplus}
    Suppose $a,b \in \reals$.
    Suppose $a < b$.
    Then $(b - a) \in \realsplus$.
\end{lemma}

\begin{lemma}\label{existence_of_epsilon_upper_border}
    Suppose $a,b \in \reals$.
    Suppose $a < b$.
    Then there exists $\epsilon \in \realsplus$ such that $b \notin \epsBall{a}{\epsilon}$. 
\end{lemma}
\begin{proof}
    Let $\epsilon = b - a$.
    Then $\epsilon \in \realsplus$.
    It suffices to show that $b \notin \epsBall{a}{\epsilon}$ by \cref{epsilon_ball,reals_addition_minus_behavior2,realsplus,minus,intervalopen,order_reals_lemma0}.
    Suppose not.
    Then $ b \in \epsBall{a}{\epsilon}$.
    Therefore $ (a - \epsilon) < b < (a + \epsilon)$.
    $b = (a + \epsilon)$.
    Contradiction.
\end{proof}

\begin{lemma}\label{existence_of_epsilon_lower_border}
    Suppose $a,b \in \reals$.
    Suppose $a > b$.
    Then there exists $\epsilon \in \realsplus$ such that $b \notin \epsBall{a}{\epsilon}$. 
\end{lemma}
\begin{proof}
    Let $\epsilon = a - b$.
    Then $\epsilon \in \realsplus$.
    It suffices to show that $b \notin \epsBall{a}{\epsilon}$ by \cref{epsilon_ball,reals_addition_minus_behavior2,realsplus,minus,intervalopen,order_reals_lemma0}.
    Suppose not.
    Then $ b \in \epsBall{a}{\epsilon}$.
    Therefore $ (a - \epsilon) < b < (a + \epsilon)$.
    $b = (a - \epsilon)$.
    Contradiction.
\end{proof}

\begin{proposition}\label{openinterval_infinite_left_in_opens}
    Suppose $a \in \reals$.
    Then $\intervalopenInfiniteLeft{a} \in \opens[\reals]$.
\end{proposition}
\begin{proof}
    Let $E = \{ B \in \pow{\reals} \mid \exists x \in \intervalopenInfiniteLeft{a} . \exists \delta \in \realsplus . B = \epsBall{x}{\delta} \land a \notin \epsBall{x}{\delta}  \}$.
    We show that for all $x \in \intervalopenInfiniteLeft{a}$ we have there exists $e \in E$ such that $x \in e$.
    \begin{subproof}
        Fix $x \in \intervalopenInfiniteLeft{a}$.
        Then $x < a$.
        Take $\delta' \in \realsplus$ such that $a \notin \epsBall{x}{\delta'}$.
        $x \in \epsBall{x}{\delta'}$ by \cref{intervalopen,epsilon_ball,reals_addition_minus_behavior1,reals_order_minus_positiv,minus,reals_add,realsplus,intervalopen_infinite_left}.
        $a \notin \epsBall{x}{\delta'}$.
        $\epsBall{x}{\delta'} \in E$.
    \end{subproof}
    $E \subseteq \topoBasisReals$.
    We show that $\unions{E} = \intervalopenInfiniteLeft{a}$.
    \begin{subproof}
        We show that $\unions{E} \subseteq \intervalopenInfiniteLeft{a}$.
        \begin{subproof}
            It suffices to show that for all $x \in \unions{E}$ we have $x \in \intervalopenInfiniteLeft{a}$.
            Fix $x \in \unions{E}$.
            Take $e \in E$ such that $x \in e$ by \cref{unions_iff}.
            $x \in \reals$.
            Take $x',\delta'$ such that $x' \in \reals$ and $\delta' \in \realsplus$ and $e = \epsBall{x'}{\delta'}$ by \cref{epsilon_ball,minus,topo_basis_reals_eps_iff,setminus,setminus_emptyset,elem_subseteq}.
            $\epsBall{x'}{\delta'} \in E$.
            We show that for all $y \in e$ we have $y < a$.
            \begin{subproof}
                Fix $y \in e$.
                Then $y \in \epsBall{x'}{\delta'}$.
                $e = \epsBall{x'}{\delta'}$.
                There exists $x'' \in \intervalopenInfiniteLeft{a}$ such that there exists $\delta'' \in \realsplus$ such that $e = \epsBall{x''}{\delta''}$ and $a \notin \epsBall{x''}{\delta''}$.
                Take $x'',\delta''$ such that $x'' \in \intervalopenInfiniteLeft{a}$ and $\delta'' \in \realsplus$ and $e = \epsBall{x''}{\delta''}$ and $a \notin \epsBall{x''}{\delta''}$.
                Suppose not.
                Take $y' \in e$ such that $y' > a$. 
                $x'' < a$.
                $(x'' - \delta'') < y' < (x'' + \delta'')$ by \cref{minus,intervalopen,epsilon_ball,realsplus,reals_add,reals_addition_minus_behavior1,reals_order_minus_positiv}.
                $(x'' - \delta'') < x'' < (x'' + \delta'')$ by \cref{minus,intervalopen,epsilon_ball,realsplus,reals_add,reals_addition_minus_behavior1,reals_order_minus_positiv}.
                Then $x'' < a < y'$.
                Therefore $(x'' - \delta'') < a < (x'' + \delta'')$ by \cref{realspuls_in_reals_minus,intervalopen_infinite_left,reals_order_is_transitive,reals_add,realsplus_in_reals,powerset_elim,subseteq}.
                Then $a \in e$ by \cref{epsball_are_connected_in_reals,intervalopen_infinite_left,neq_witness}.
                Contradiction.
            \end{subproof}
            $x < a$.
            Then $x \in \intervalopenInfiniteLeft{a}$.
        \end{subproof}
        We show that $\intervalopenInfiniteLeft{a} \subseteq \unions{E}$.
        \begin{subproof}
            Trivial.
        \end{subproof}
    \end{subproof}
    $\unions{E} \in \opens[\reals]$ by \cref{opens_unions,reals_is_topological_space,basis_is_in_genopens,topological_space_reals,topological_basis_reals_is_basis,subset_transitive}.
\end{proof}

\begin{lemma}\label{continuous_on_basis_implies_continuous_endo}
    Suppose $X$ is a topological space.
    Suppose $B$ is a topological basis for $X$.
    Suppose $f$ is a function from $X$ to $X$.
    $f$ is continuous iff for all $b \in B$ we have $\preimg{f}{b} \in \opens[X]$.
\end{lemma}
\begin{proof}
    Omitted.
\end{proof}

\begin{proposition}\label{openinterval_infinite_right_in_opens}
    Suppose $a \in \reals$.
    Then $\intervalopenInfiniteRight{a} \in \opens[\reals]$.
\end{proposition}
\begin{proof}
    Let $E = \{ B \in \pow{\reals} \mid \exists x \in \intervalopenInfiniteRight{a} . \exists \delta \in \realsplus . B = \epsBall{x}{\delta} \land a \notin \epsBall{x}{\delta}  \}$.
    We show that for all $x \in \intervalopenInfiniteRight{a}$ we have there exists $e \in E$ such that $x \in e$.
    \begin{subproof}
        Fix $x \in \intervalopenInfiniteRight{a}$.
        Then $a < x$.
        Take $\delta' \in \realsplus$ such that $a \notin \epsBall{x}{\delta'}$.
        $x \in \epsBall{x}{\delta'}$ by \cref{intervalopen,epsilon_ball,reals_addition_minus_behavior1,reals_order_minus_positiv,minus,reals_add,realsplus,intervalopen_infinite_right}.
        $a \notin \epsBall{x}{\delta'}$.
        $\epsBall{x}{\delta'} \in E$.
    \end{subproof}
    $E \subseteq \topoBasisReals$.
    We show that $\unions{E} = \intervalopenInfiniteRight{a}$.
    \begin{subproof}
        We show that $\unions{E} \subseteq \intervalopenInfiniteRight{a}$.
        \begin{subproof}
            It suffices to show that for all $x \in \unions{E}$ we have $x \in \intervalopenInfiniteRight{a}$.
            Fix $x \in \unions{E}$.
            Take $e \in E$ such that $x \in e$ by \cref{unions_iff}.
            $x \in \reals$.
            Take $x',\delta'$ such that $x' \in \reals$ and $\delta' \in \realsplus$ and $e = \epsBall{x'}{\delta'}$ by \cref{epsilon_ball,minus,topo_basis_reals_eps_iff,setminus,setminus_emptyset,elem_subseteq}.
            $\epsBall{x'}{\delta'} \in E$.
            We show that for all $y \in e$ we have $y > a$.
            \begin{subproof}
                Fix $y \in e$.
                Then $y \in \epsBall{x'}{\delta'}$.
                $e = \epsBall{x'}{\delta'}$.
                There exists $x'' \in \intervalopenInfiniteRight{a}$ such that there exists $\delta'' \in \realsplus$ such that $e = \epsBall{x''}{\delta''}$ and $a \notin \epsBall{x''}{\delta''}$.
                Take $x'',\delta''$ such that $x'' \in \intervalopenInfiniteRight{a}$ and $\delta'' \in \realsplus$ and $e = \epsBall{x''}{\delta''}$ and $a \notin \epsBall{x''}{\delta''}$.
                Suppose not.
                Take $y' \in e$ such that $y' < a$ by \cref{reals_order_total,intervalopen,eps_ball_implies_open_interval}. 
                $x'' > a$.
                $(x'' - \delta'') < y' < (x'' + \delta'')$ by \cref{minus,intervalopen,intervalclosed,epsilon_ball,realsplus,reals_add,reals_addition_minus_behavior1,reals_order_minus_positiv}.
                $(x'' - \delta'') < x'' < (x'' + \delta'')$ by \cref{minus,intervalopen,intervalclosed,epsilon_ball,realsplus,reals_add,reals_addition_minus_behavior1,reals_order_minus_positiv}.
                Then $x'' > a > y'$.
                Therefore $(x'' - \delta'') > a > (x'' + \delta'')$ by \cref{realspuls_in_reals_minus,intervalopen_infinite_right,reals_order_is_transitive,reals_add,realsplus_in_reals,powerset_elim,subseteq,epsball_are_connected_in_reals,subseteq}.
                Then $a \in e$ by \cref{reals_order_is_transitive,reals_order_total,reals_add,realsplus,epsball_are_connected_in_reals,intervalopen_infinite_right,neq_witness}.
                Contradiction.
            \end{subproof}
            $x > a$.
            Then $x \in \intervalopenInfiniteRight{a}$.
        \end{subproof}
        We show that $\intervalopenInfiniteRight{a} \subseteq \unions{E}$.
        \begin{subproof}
            Trivial.
        \end{subproof}
    \end{subproof}
    $\unions{E} \in \opens[\reals]$.

    %Let $I = \{\neg{b} \mid b \in \intervalopenInfiniteRight{a} \}$.
    %Let $f(x) = \neg{x}$ for $x \in \reals$.
    %$f$ is a function from $\reals$ to $\reals$.
    %We show that $f$ is continuous.
    %\begin{subproof}
    %    It suffices to show that for all $b \in \topoBasisReals$ we have $\preimg{f}{b} \in \opens[\reals]$.
    %    Fix $b \in \topoBasisReals$.
    %    Take $x, \epsilon$ such that $x \in \reals$ and $\epsilon \in \realsplus$ and $b = \epsBall{x}{\epsilon}$.
    %    Let $y = \neg{x}$.
    %    It suffices to show that $\preimg{f}{b} \in \topoBasisReals$ by \cref{topological_space_reals,topological_basis_reals_is_basis,basis_is_in_genopens,cons_remove,cons_subseteq_iff}.
    %    It suffices to show that $\epsBall{y}{\epsilon} = \preimg{f}{\epsBall{x}{\epsilon}}$.
    %    $\preimg{f}{\epsBall{x}{\epsilon}} \subseteq \reals$.
    %    $\epsBall{y}{\epsilon} \subseteq \reals$ by \cref{intervalopen,subseteq,minus,epsilon_ball}.
    %    %It suffices to show that for all $x \in \reals$ we have $x \in \epsBall{y}{\epsilon}$ iff $x \in \preimg{f}{\epsBall{x}{\epsilon}}$.
    %    %Fix $x \in \reals$.
    %    Let $u = (y - \epsilon)$.
    %    Let $v = (y + \epsilon)$.
    %    $u = \neg{(x - \epsilon)}$.
    %    $v = \neg{(x + \epsilon)}$.
    %    %$v - u = \epsilon + \epsilon$.
    %    We show that for all $z \in \epsBall{y}{\epsilon}$ we have $z \in \preimg{f}{\epsBall{x}{\epsilon}}$.
    %    \begin{subproof}
    %        Fix $z \in \epsBall{y}{\epsilon}$.
    %        Then $u < z < v$.
    %        Let $z' = z - u$.
    %        Then $z = u + z'$.
    %        Suppose not.
    %        Let $h = \neg{z}$.
    %        $\neg{h} = \neg{\neg{z}}$.
    %        $\neg{h} = z$.
    %        Then $f(h) = \neg{h}$.
    %        $f(h) = z$.
    %        Then $z \in \preimg{f}{\{h\}}$.
%
    %    \end{subproof}
    %    We show that for all $z \in \preimg{f}{\epsBall{x}{\epsilon}}$ we have $z \in \epsBall{y}{\epsilon}$.
    %    \begin{subproof}
    %        Fix $z \in \preimg{f}{\epsBall{x}{\epsilon}}$.
    %        Take $h \in \epsBall{x}{\epsilon}$ such that $f(h) = z$.
    %    \end{subproof}
    %    Follows by set extensionality.
    %\end{subproof}
    %$\intervalopenInfiniteLeft{a} \in \opens[\reals]$.
    %We show that $\preimg{f}{\intervalopenInfiniteLeft{a}} = \intervalopenInfiniteRight{a}$.
    %\begin{subproof}
    %    Omitted.
    %\end{subproof}
    %Then $\intervalopenInfiniteRight{a} \in \opens[\reals]$ by \cref{continuous,preim_eq_img_of_converse,openinterval_infinite_left_in_opens}.
\end{proof}

\begin{lemma}\label{reals_as_union_of_open_closed_intervals1}
    Suppose $a,b \in \reals$.
    Then $\reals = \intervalopenInfiniteLeft{a} \union \intervalopenInfiniteRight{b} \union \intervalclosed{a}{b}$.
\end{lemma}
\begin{proof}
    We show that for all $x \in \reals$ we have $x \in (\intervalopenInfiniteLeft{a} \union \intervalopenInfiniteRight{b} \union \intervalclosed{a}{b})$.
    \begin{subproof}
        Fix $x \in \reals$.
        Follows by \cref{union_intro_left,intervalopen_infinite_left,reals_order_total,reals_order_total2,union_iff,intervalopen_infinite_right,union_assoc,union_intro_right,intervalclosed}.
    \end{subproof}
\end{proof}

\begin{lemma}\label{reals_as_union_of_open_closed_intervals2}
    Suppose $a \in \reals$.
    Then $\reals = \intervalopenInfiniteLeft{a} \union \intervalclosedInfiniteRight{a}$.
\end{lemma}
\begin{proof}
    It suffices to show that for all $x \in \reals$ we have either $x \in \intervalopenInfiniteLeft{a}$ or $x \in \intervalclosedInfiniteRight{a}$ by \cref{intervalopen_infinite_left,union_intro_left,neq_witness,intervalclosed_infinite_right,union_intro_right,union_iff}.
    Trivial.
\end{proof}

\begin{lemma}\label{reals_as_union_of_open_closed_intervals3}
    Suppose $a \in \reals$.
    Then $\reals = \intervalopenInfiniteRight{a} \union \intervalclosedInfiniteLeft{a}$.
\end{lemma}
\begin{proof}
    It suffices to show that for all $x \in \reals$ we have either $x \in \intervalclosedInfiniteLeft{a}$ or $x \in \intervalopenInfiniteRight{a}$.
    Trivial.
\end{proof}

\begin{lemma}\label{intersection_of_open_closed__infinite_intervals_open_right}
    Suppose $a \in \reals$.
    Then $\intervalopenInfiniteRight{a} \inter \intervalclosedInfiniteLeft{a} = \emptyset$.
\end{lemma}
\begin{proof}
    Follows by \cref{reals_order_total,inter_lower_left,intervalopen_infinite_right,order_reals_lemma6,inter_lower_right,foundation,subseteq,intervalclosed_infinite_left}.
\end{proof}

\begin{lemma}\label{intersection_of_open_closed__infinite_intervals_open_left}
    Suppose $a \in \reals$.
    Then $\intervalopenInfiniteLeft{a} \inter \intervalclosedInfiniteRight{a} = \emptyset$.
\end{lemma}

\begin{proposition}\label{closedinterval_infinite_right_in_closeds}
    Suppose $a \in \reals$.
    Then $\intervalclosedInfiniteRight{a} \in \closeds{\reals}$.
\end{proposition}
\begin{proof}
    $\intervalclosedInfiniteRight{a} = \reals \setminus \intervalopenInfiniteLeft{a}$ by \cref{intersection_of_open_closed__infinite_intervals_open_left,reals_as_union_of_open_closed_intervals2,setminus_inter,double_relative_complement,subseteq_union_setminus,subseteq_setminus,setminus_union,setminus_disjoint,setminus_partition,setminus_subseteq,setminus_emptyset,setminus_self,setminus_setminus,double_complement_union}.
\end{proof}

\begin{proposition}\label{closedinterval_infinite_left_in_closeds}
    Suppose $a \in \reals$.
    Then $\intervalclosedInfiniteLeft{a} \in \closeds{\reals}$.
\end{proposition}
\begin{proof}
    $\intervalclosedInfiniteLeft{a} = \reals \setminus \intervalopenInfiniteRight{a}$.
\end{proof}

\begin{proposition}\label{closedinterval_eq_openintervals_setminus_reals}
    Suppose $a,b \in \reals$.
    Then $\reals \setminus (\intervalopenInfiniteLeft{a} \union \intervalopenInfiniteRight{b}) = \intervalclosed{a}{b}$.
\end{proposition}
\begin{proof}
    We have $\intervalclosed{a}{b} \subseteq \reals$.
    We show that $\reals \setminus (\intervalopenInfiniteLeft{a} \union \intervalopenInfiniteRight{b}) = (\reals \setminus \intervalopenInfiniteLeft{a}) \inter (\reals \setminus \intervalopenInfiniteRight{b})$.
    \begin{subproof}
        We show that for all $x \in \reals \setminus (\intervalopenInfiniteLeft{a} \union \intervalopenInfiniteRight{b})$ we have $x \in (\reals \setminus \intervalopenInfiniteLeft{a}) \inter (\reals \setminus \intervalopenInfiniteRight{b})$.
        \begin{subproof}
            Fix $x \in \reals \setminus (\intervalopenInfiniteLeft{a} \union \intervalopenInfiniteRight{b})$.
            Then $x \in \reals \setminus \intervalopenInfiniteLeft{a}$ by \cref{setminus,double_complement_union}.
            Then $x \in \reals \setminus \intervalopenInfiniteRight{b}$ by \cref{union_upper_left,subseteq,union_comm,subseteq_implies_setminus_supseteq}.
            Follows by \cref{inter_intro}.
        \end{subproof}
        We show that for all $x \in (\reals \setminus \intervalopenInfiniteLeft{a}) \inter (\reals \setminus \intervalopenInfiniteRight{b})$ we have $x \in \reals \setminus (\intervalopenInfiniteLeft{a} \union \intervalopenInfiniteRight{b})$.
        \begin{subproof}
            Fix $x \in (\reals \setminus \intervalopenInfiniteLeft{a}) \inter (\reals \setminus \intervalopenInfiniteRight{b})$.
            Then $x \in (\reals \setminus \intervalopenInfiniteLeft{a})$ by \cref{setminus_setminus,setminus}.
            Then $x \in (\reals \setminus \intervalopenInfiniteRight{b})$ by \cref{inter_lower_right,elem_subseteq,setminus_setminus}.
        \end{subproof}
        Follows by \cref{setminus_union}.
    \end{subproof}
    We show that $\reals \setminus \intervalopenInfiniteLeft{a} = \intervalclosedInfiniteRight{a}$.
    \begin{subproof}
        For all $x \in \intervalclosedInfiniteRight{a}$ we have $x \geq a$.
        For all $x \in (\reals \setminus \intervalopenInfiniteLeft{a})$ we have $x \geq a$.
        Follows by set extensionality.
    \end{subproof}
    $\reals \setminus \intervalopenInfiniteRight{b} = \intervalclosedInfiniteLeft{b}$.
    It suffices to show that $\intervalclosedInfiniteLeft{b} \inter \intervalclosedInfiniteRight{a} = \intervalclosed{a}{b}$.
    For all $x \in \intervalclosed{a}{b}$ we have $a \leq x \leq b$.
    For all $x \in (\intervalclosedInfiniteLeft{b} \inter \intervalclosedInfiniteRight{a})$ we have $a \leq x \leq b$.
    Follows by set extensionality.
\end{proof}

\begin{proposition}\label{closedinterval_is_closed}
    Suppose $a,b \in \reals$.
    Then $\intervalclosed{a}{b} \in \closeds{\reals}$.
\end{proposition}
\begin{proof}
    We have $\reals = \intervalopenInfiniteLeft{a} \union \intervalopenInfiniteRight{b} \union \intervalclosed{a}{b}$.
    It suffices to show that $\reals \setminus (\intervalopenInfiniteLeft{a} \union \intervalopenInfiniteRight{b}) = \intervalclosed{a}{b}$ by \cref{closeds,setminus_subseteq,powerset_intro,closed_minus_open_is_closed,opens_type,subseteq_refl,union_open,is_closed_in,reals_carrier_reals,setminus_self,emptyset_open,reals_is_topological_space,openinterval_infinite_left_in_opens,openinterval_infinite_right_in_opens}.
\end{proof}
