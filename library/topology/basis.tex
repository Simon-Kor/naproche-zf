\import{topology/topological-space.tex}

\subsection{Topological basis}

\begin{abbreviation}\label{covers}
    $C$ covers $X$ iff
    for all $x\in X$ there exists $U\in C$ such that $x\in U$.
\end{abbreviation}

\begin{proposition}\label{covers_unions_intro}
    Suppose $C$ covers $X$.
    Then $X\subseteq\unions{C}$.
\end{proposition}

\begin{proposition}\label{covers_unions_elim}
    Suppose $X\subseteq\unions{C}$.
    Then $C$ covers $X$.
\end{proposition}

% Also called "prebase", "subbasis", or "subbase". We prefer "pre-" or "quasi-"
% for consistency when handling generalizations, even if "subbasis" is more common.
\begin{abbreviation}\label{topological_prebasis}
    $B$ is a topological prebasis for $X$ iff $\unions{B} = X$.
\end{abbreviation}

\begin{proposition}\label{topological_prebasis_iff_covering_family}
    $B$ is a topological prebasis for $X$ iff
    $B$ is a family of subsets of $X$ and $B$ covers $X$.
\end{proposition}
\begin{proof}
    If $B$ is a family of subsets of $X$ and $B$ covers $X$,
        then $\unions{B} = X$
            by \cref{subseteq_antisymmetric,unions_family,covers_unions_intro}.
    If $\unions{B} = X$,
        then $B$ is a family of subsets of $X$ and $B$ covers $X$
            by \cref{covers_unions_intro,subseteq_refl,covers_unions_elim}.
\end{proof}

% Also called "base of topology".
\begin{definition}\label{topological_basis}
    $B$ is a topological basis for $X$ iff
    $B$ is a topological prebasis for $X$ and
    for all $U, V, x$ such that $U, V\in B$ and $x\in U,V$
    there exists $W\in B$ such that $x\in W\subseteq U, V$.
\end{definition}

\begin{definition}\label{genOpens}
    $\genOpens{B}{X} = \{ U\in\pow{X} \mid for all $x\in U$ there exists $V\in B$
    such that $x\in V\subseteq U$\}$.
\end{definition}
