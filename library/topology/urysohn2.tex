\import{topology/topological-space.tex}
\import{topology/separation.tex}
\import{topology/continuous.tex}
\import{topology/basis.tex}
\import{numbers.tex}
\import{function.tex}
\import{set.tex}
\import{cardinal.tex}
\import{relation.tex}
\import{relation/uniqueness.tex}
\import{set/cons.tex}
\import{set/powerset.tex}
\import{set/fixpoint.tex}
\import{set/product.tex}
\import{topology/real-topological-space.tex}
\import{set/equinumerosity.tex}

\section{Urysohns Lemma}




\begin{definition}\label{sequence}
    $X$ is a sequence iff $X$ is a function and $\dom{X} \subseteq \naturals$.
\end{definition}


\begin{abbreviation}\label{urysohnspace}
    $X$ is a urysohn space iff
    $X$ is a topological space and
    for all $A,B \in \closeds{X}$ such that $A \inter B = \emptyset$
    we have there exist $A',B' \in \opens[X]$
    such that  $A \subseteq A'$ and $B \subseteq B'$ and $A' \inter B' = \emptyset$.    
\end{abbreviation}


\begin{abbreviation}\label{at}
    $\at{f}{n} = f(n)$.
\end{abbreviation}


\begin{definition}\label{chain_of_subsets}
    $X$ is a chain of subsets in $Y$ iff $X$ is a sequence and for all $n \in \dom{X}$ we have $\at{X}{n} \subseteq \carrier[Y]$ and for all $m \in \dom{X}$ such that $m > n$ we have $\at{X}{n} \subseteq \at{X}{m}$. 
\end{definition}


\begin{definition}\label{urysohnchain}%<-- zulässig
    $X$ is a urysohnchain of $Y$ iff $X$ is a chain of subsets in $Y$ and for all $n,m \in \dom{X}$ such that $n < m$ we have $\closure{\at{X}{n}}{Y} \subseteq \interior{\at{X}{m}}{Y}$.
\end{definition}

\begin{definition}\label{urysohn_finer_set}
    $A$ is finer between $X$ to $Y$ in $U$ iff $\closure{X}{U} \subseteq \interior{A}{U}$ and $\closure{A}{U} \subseteq \interior{Y}{U}$.
\end{definition}

\begin{definition}\label{finer} %<-- verfeinerung 
    $Y$ is finer then $X$ in $U$ iff for all $n \in \dom{X}$ we have $\at{X}{n} \in \ran{Y}$ and for all $m \in \dom{X}$ such that $n < m$ we have there exist $k \in \dom{Y}$ such that $\at{Y}{k}$ is finer between $\at{X}{n}$ to $\at{X}{m}$ in $U$.
\end{definition}

\begin{definition}\label{follower_index}
    $y$ follows $x$ in $I$ iff $x < y$ and $x,y \in I$ and for all $i \in I$ such that $x < i$ we have $y \leq i$.
\end{definition}

\begin{definition}\label{finer_smallest_step}
    $Y$ is a minimal finer extention of $X$ in $U$ iff $Y$ is finer then $X$ in $U$ and for all $x_1,x_2 \in \dom{X}$ such that $x_1$ follows $x_2$ in $\dom{X}$ we have there exist $y \in \dom{Y}$ such that $y$ follows $x_1$ in $\dom{X}$ and $x_2$ follows $y$ in $\dom{X}$.
\end{definition}

\begin{definition}\label{sequence_of_reals}
    $X$ is a sequence of reals iff $\ran{X} \subseteq \reals$.
\end{definition}


\begin{definition}\label{pointwise_convergence}
    $X$ converge to $x$ iff for all $\epsilon \in \realsplus$ there exist $N \in \dom{X}$ such that for all $n \in \dom{X}$ such that $n > N$ we have $\at{X}{n} \in \epsBall{x}{\epsilon}$.
\end{definition}


\begin{proposition}\label{iff_sequence}
    Suppose $X$ is a function.
    Suppose $\dom{X} \subseteq \naturals$.
    Then $X$ is a sequence.
\end{proposition}

\begin{definition}\label{lifted_urysohn_chain}
    $U$ is a lifted urysohnchain of $X$ iff $U$ is a sequence and $\dom{U} = \naturals$ and for all $n \in \dom{U}$ we have $\at{U}{n}$ is a urysohnchain of $X$ and $\at{U}{\suc{n}}$ is a minimal finer extention of $\at{U}{n}$ in $X$.
\end{definition}

\begin{definition}\label{normal_ordered_urysohnchain}
    $U$ is normal ordered iff there exist $n \in \naturals$ such that $\dom{U} = \seq{\zero}{n}$.
\end{definition}

\begin{definition}\label{bijection_of_urysohnchains}
    $f$ is consistent on $X$ to $Y$ iff $f$ is a bijection from $\dom{X}$ to $\dom{Y}$ and for all $n,m \in \dom{X}$ such that $n < m$ we have $f(n) < f(m)$.
\end{definition}

\begin{proposition}\label{naturals_in_transitive}
    $\naturals$ is a \in-transitive set.
\end{proposition}
\begin{proof}
    Follows by \cref{nat_is_transitiveset}.
\end{proof}

\begin{proposition}\label{naturals_elem_in_transitive}
    If $n \in \naturals$ then $n$ is \in-transitive and every element of $n$ is \in-transitive. 
    %If $n$ is a natural number then $n$ is \in-transitive and every element of $n$ is \in-transitive. 
\end{proposition}

\begin{proposition}\label{natural_number_is_ordinal_for_all}
    For all $n \in \naturals$ we have $n$ is a ordinal.
\end{proposition}

\begin{proposition}\label{zero_is_in_minimal}
    $\zero$ is an \in-minimal element of $\naturals$.
\end{proposition}

\begin{proposition}\label{natural_rless_eq_precedes}
    For all $n,m \in \naturals$ we have $n \precedes m$ iff $n \in m$.
\end{proposition}

\begin{proposition}\label{naturals_precedes_suc}
    For all $n \in \naturals$ we have $n \precedes \suc{n}$.
\end{proposition}

\begin{proposition}\label{zero_is_empty}
    There exists no $x$ such that $x \in \zero$.
\end{proposition}
\begin{proof}
    Follows by \cref{notin_emptyset}.
\end{proof}

\begin{proposition}\label{one_is_positiv}
    $1$ is positiv.
\end{proposition}

\begin{proposition}\label{suc_of_positive_is_positive}
    For all $n \in \naturals$ such that $n$ is positiv we have $\suc{n}$ is positiv.
\end{proposition}

\begin{proposition}\label{naturals_are_positiv_besides_zero}
    For all $n \in \naturals$ such that $n \neq \zero$ we have $n$ is positiv.
\end{proposition}
\begin{proof}[Proof by \in-induction on $n$]
    Assume $n \in \naturals$.
    \begin{byCase}
        \caseOf{$n = \zero$.} Trivial.
        \caseOf{$n \neq \zero$.}
            Take $k \in \naturals$ such that $\suc{k} = n$.
    \end{byCase}
\end{proof}



\begin{proposition}\label{naturals_sum_eq_zero}
    For all $n,m \in \naturals$ we have if $n+m = \zero$ then $n = m = \zero$.
\end{proposition}
\begin{proof}
    Omitted.
\end{proof}

\begin{proposition}\label{no_natural_between_n_and_suc_n}
    For all $n,m \in \naturals$ we have not $n < m < \suc{n}$.
\end{proposition}
\begin{proof}
    Omitted.
\end{proof}

\begin{proposition}\label{naturals_is_zero_one_or_greater}
    $\naturals = \{n \in \naturals \mid n > 1 \lor n = 1 \lor n = \zero\}$.
\end{proposition}
\begin{proof}
    Omitted.
\end{proof}

\begin{proposition}\label{naturals_one_zero_or_greater}
    For all $l \in \naturals$ we have if $l < 1$ then $l = \zero$.
\end{proposition}
\begin{proof}
    Follows by \cref{reals_order,naturals_subseteq_reals,subseteq,one_in_reals,naturals_is_zero_one_or_greater}.
\end{proof}

\begin{proposition}\label{naturals_rless_existence_of_lesser_natural}
    For all $n \in \naturals$ we have for all $m \in \naturals$ such that $m < n$ there exist $k \in \naturals$ such that $m + k = n$.
\end{proposition}
\begin{proof}[Proof by \in-induction on $n$]
    Assume $n \in \naturals$.
    
    \begin{byCase}
        \caseOf{$n = \zero$.} 
            
            We show that for all $m \in \naturals$ such that $m < n$ we have there exist $k \in \naturals$ such that $m + k = n$.
            \begin{subproof}[Proof by \in-induction on $m$]
                Assume $m \in \naturals$.
                \begin{byCase}
                    \caseOf{$m = \zero$.}
                        Trivial.
                    \caseOf{$m \neq \zero$.}
                        Trivial.
                \end{byCase}
            \end{subproof}
        \caseOf{$n = 1$.}
            Fix $m$.
            For all $l \in \naturals$ we have if $l < 1$ then $l = \zero$.
            Then $\zero + 1 = 1$.
        \caseOf{$n > 1$.} 
            Take $l \in \naturals$ such that $\suc{l} = n$.
            Omitted.
    \end{byCase}
\end{proof}


\begin{proposition}\label{rless_eq_in_for_naturals}
    For all $n,m \in \naturals$ such that $n < m$ we have $n \in m$. 
\end{proposition}
\begin{proof}
    We show that for all $n \in \naturals$ we have for all $m \in \naturals$ such that $m < n$ we have $m \in n$.
    \begin{subproof}[Proof by \in-induction on $n$]
        Assume $n \in \naturals$.
        We show that for all $m \in \naturals$ such that$m < n$ we have $m \in n$.
        \begin{subproof}[Proof by \in-induction on $m$]
            Assume $m \in \naturals$.
            \begin{byCase}
                \caseOf{$\suc{m}=n$.}
                \caseOf{$\suc{m}\neq n$.}  
                      \begin{byCase}
                        \caseOf{$n = \zero$.}
                        \caseOf{$n \neq \zero$.}
                            Take $l \in \naturals$ such that $\suc{l} = n$.
                            Omitted.

                      \end{byCase}
            \end{byCase}
        \end{subproof}
    \end{subproof}
    
    %Fix $n \in \naturals$.

    %\begin{byCase}
    %    \caseOf{$n = \zero$.}
    %        For all $k \in \naturals$ we have $k = \zero$ or $\zero < k$.
    %        
    %    \caseOf{$n \neq \zero$.}
    %        Fix $m \in \naturals$.
    %        It suffices to show that $m \in n$.
    %\end{byCase}
    
\end{proof}



\begin{proposition}\label{naturals_leq}
    For all $n \in \naturals$ we have $\zero \leq n$.
\end{proposition}




\begin{proposition}\label{naturals_leq_on_suc}
    For all $n,m \in \naturals$ such that $m < \suc{n}$ we have $m \leq n$.
\end{proposition}
\begin{proof}
    Omitted.
\end{proof}

\begin{proposition}\label{x_in_seq_iff}
    Suppose $n,m,x \in \naturals$.
    $x \in \seq{n}{m}$ iff $n \leq x \leq m$.
\end{proposition}

\begin{proposition}\label{seq_zero_to_n_eq_to_suc_n}
    For all $n \in \naturals$ we have $\seq{\zero}{n} = \suc{n}$.
\end{proposition}
\begin{proof} [Proof by \in-induction on $n$]
    Assume $n \in \naturals$.
    $n \in \naturals$.
    For all $m \in n$ we have $m \in \naturals$.
    \begin{byCase}
        \caseOf{$n = \zero$.}
            It suffices to show that $1 = \seq{\zero}{\zero}$.
            Follows by set extensionality.
        \caseOf{$n \neq \zero$.}
            Take $k$ such that $k \in \naturals$ and $\suc{k} = n$.
            Then $k \in n$.
            Therefore $\seq{\zero}{k} = \suc{k}$.
            We show that $\seq{\zero}{n} = \seq{\zero}{k} \union \{n\}$.
            \begin{subproof}
                We show that $\seq{\zero}{n} \subseteq \seq{\zero}{k} \union \{n\}$.
                \begin{subproof}
                    It suffices to show that for all $x \in \seq{\zero}{n}$ we have $x \in \seq{\zero}{k} \union \{n\}$.
                    $n \in \naturals$.
                    $\zero \leq n$.
                    $n \leq n$.
                    We have $n \in \seq{\zero}{n}$.
                    Therefore $\seq{\zero}{n}$ is inhabited.
                    Take $x$ such that $x \in \seq{\zero}{n}$.
                    Therefore $\zero \leq x \leq n$.
                    $x = n$ or $x < n$.
                    Then either $x = n$ or $x \leq k$.
                    Therefore $x \in \seq{\zero}{k}$ or $x = n$.
                    Follows by \cref{reals_order,natural_number_is_ordinal,ordinal_empty_or_emptyset_elem,naturals_leq_on_suc,reals_axiom_zero_in_reals,naturals_subseteq_reals,subseteq,union_intro_left,naturals_inductive_set,m_to_n_set,x_in_seq_iff,union_intro_right,singleton_intro}.
                \end{subproof}
                We show that $\seq{\zero}{k} \union \{n\} \subseteq \seq{\zero}{n}$.
                \begin{subproof}
                    It suffices to show that for all $x \in \seq{\zero}{k} \union \{n\}$ we have $x \in \seq{\zero}{n}$.
                    $k \leq n$ by \cref{naturals_subseteq_reals,suc_eq_plus_one,plus_one_order,subseteq}.
                    $k \in n$.
                    $\seq{\zero}{k} = \suc{k}$ by assumption.            
                    $n \in \naturals$.
                    $\zero \leq n$.
                    $n \leq n$.
                    We have $n \in \seq{\zero}{n}$.
                    Therefore $\seq{\zero}{n}$ is inhabited.
                    Take $x$ such that $x \in \seq{\zero}{n}$.
                    Therefore $\zero \leq x \leq n$.
                    $x = n$ or $x < n$.
                    Then either $x = n$ or $x \leq k$.
                    Therefore $x \in \seq{\zero}{k}$ or $x = n$.
                    Fix $x$.
                    \begin{byCase}
                        \caseOf{$x \in \seq{\zero}{k}$.}
                            Trivial.
                        \caseOf{$x = n$.}
                            It suffices to show that $n \in \seq{\zero}{n}$.
                    \end{byCase}
                \end{subproof}
                Trivial.
            \end{subproof}
            We have $\suc{n} = n \union \{n\}$.
    \end{byCase}

    %Assume $n$ is a natural number.
    %We show that $\seq{\zero}{\zero}$ has cardinality $1$.
    %\begin{subproof}
    %    It suffices to show that $1 = \seq{\zero}{\zero}$.
    %    Follows by set extensionality.
    %\end{subproof}
    %It suffices to show that if $n \neq \zero$ then $\seq{\zero}{n}$ has cardinality $\suc{n}$.
    %We show that for all $m \in \naturals$ such that $m \neq \zero$ we have $\seq{\zero}{m}$ has cardinality $\suc{m}$.
    %\begin{subproof}
    %    Fix $m \in \naturals$.
    %    Take $k$ such that $k \in \naturals$ and $\suc{k} = m$.
    %    Then $k \in m$.
    %\end{subproof}
    
    
    %For all $n \in \naturals$ we have either $n = \zero$ or there exist $m \in \naturals$ such that $n = \suc{m}$.
    %For all $n \in \naturals$ we have either $n = \zero$ or $\zero \in n$.
    %We show that if $\seq{\zero}{n}$ has cardinality $\suc{n}$ then $\seq{\zero}{\suc{n}}$ has cardinality $\suc{\suc{n}}$.
    %\begin{subproof}
    %    Omitted.
    %\end{subproof}
\end{proof}

\begin{proposition}\label{seq_zero_to_n_isomorph_to_suc_n}
    For all $n \in \naturals$ we have $\seq{\zero}{n}$ has cardinality $\suc{n}$.
\end{proposition}

\begin{proposition}\label{bijection_naturals_order}
    For all $M \subseteq \naturals$ such that $M$ is inhabited we have there exist $f,k$ such that $f$ is a bijection from $\seq{\zero}{k}$ to $M$ and $M$ has cardinality $\suc{k}$ and for all $n,m \in \seq{\zero}{k}$ such that $n < m$ we have $f(n) < f(m)$.
\end{proposition}
\begin{proof}
    Omitted.
\end{proof}

\begin{lemma}\label{naturals_suc_injective}
    Suppose $n,m \in \naturals$.
    $n = m$ iff $\suc{n} = \suc{m}$.
\end{lemma}

\begin{lemma}\label{naturals_rless_implies_not_eq}
    Suppose $n,m \in \naturals$.
    Suppose $n < m$.
    Then $n \neq m$.
\end{lemma}

\begin{lemma}\label{cardinality_of_singleton}
    For all $x$ such that $x \neq \emptyset$ we have $\{x\}$ has cardinality $1$.
\end{lemma}
\begin{proof}
    Omitted.
    %Fix $x$.
    %Suppose $x \neq \emptyset$.
    %Let $X = \{x\}$.
    %$\seq{\zero}{\zero}=1$.
    %$\seq{\zero}{\zero}$ has cardinality $1$.
    %$X \setminus \{x\} = \emptyset$.
    %$1 = \{\emptyset\}$.
    %Let $F = \{(x,\emptyset)\}$.
    %$F$ is a relation.
    %$\dom{F} = X$.
    %$\emptyset \in \ran{F}$.
    %for all $x \in 1$ we have $x = \emptyset$.
    %$\ran{F} = 1$.
    %$F$ is injective.
    %$F \in \surj{X}{1}$.
    %$F$ is a bijection from $X$ to $1$.
\end{proof}

\begin{lemma}\label{cardinality_n_plus_1}
    For all $n \in \naturals$ we have $n+1$ has cardinality $n+1$.
\end{lemma}
\begin{proof}
    Omitted.
\end{proof}

\begin{lemma}\label{cardinality_n_m_plus}
    For all $n,m \in \naturals$ we have $n+m$ has cardinality $n+m$.
\end{lemma}
\begin{proof}
    Omitted.
\end{proof}

\begin{lemma}\label{cardinality_plus_disjoint}
    Suppose $X \inter Y = \emptyset$.
    Suppose $X$ is finite.
    Suppose $Y$ is finite.
    Suppose $X$ has cardinality $n$.
    Suppose $Y$ has cardinality $m$.
    Then $X \union Y$ has cardinality $m+n$.
\end{lemma}
\begin{proof}
    Omitted.
\end{proof}




\begin{lemma}\label{injective_functions_is_bijection_if_bijection_there_is_other_bijection_1}
    Suppose $f$ is a bijection from $X$ to $Y$.
    Suppose $g$ is a function from $X$ to $Y$.
    Suppose $g$ is injective.
    Suppose $X$ is finite and $Y$ is finite. 
    For all $n \in \naturals$ such that $Y$ has cardinality $n$ we have $g$ is a bijection from $X$ to $Y$.
\end{lemma}
\begin{proof}[Proof by \in-induction on $n$]
    Assume $n \in \naturals$.
    Suppose $Y$ has cardinality $n$.
    $X$ has cardinality $n$ by \cref{bijection_converse_is_bijection,bijection_circ,regularity,cardinality,foundation,empty_eq,notin_emptyset}.
    \begin{byCase}
        \caseOf{$n = \zero$.} 
            Follows by \cref{converse_converse_eq,injective_converse_is_function,converse_is_relation,dom_converse,id_is_function_to,id_ran,ran_circ_exact,circ,ran_converse,emptyset_is_function_on_emptyset,bijective_converse_are_funs,relext,function_member_elim,bijection_is_function,cardinality,bijections_dom,in_irrefl,codom_of_emptyset_can_be_anything,converse_emptyset,funs_elim,neq_witness,id}.
        \caseOf{$n \neq \zero$.}
            %Take $n' \in n$ such that $n = \suc{n'}$.
            %$n' \in \naturals$.
            %$n' + 1 = n$.
            %Take $y$ such that $y \in Y$ by \cref{funs_type_apply,apply,bijections_to_funs,cardinality,foundation}. 
            %Let $Y' = Y \setminus \{y\}$.
            %$Y' \subseteq Y$.
            %$Y'$ is finite.
            %There exist $m \in \naturals$ such that $Y'$ has cardinality $m$.
            %Take $m \in \naturals$ such that $Y'$ has cardinality $m$.
            %Then $Y'$ has cardinality $n'$.
            %Let $x' = \apply{\converse{f}}{y'}$. 
            %$x' \in X$.
            %Let $X' = X \setminus \{x'\}$.
            %$X' \subseteq X$.
            %$X'$ is finite.
            %There exist $m' \in \naturals$ such that $X'$ has cardinality $m'$.
            %Take $m' \in \naturals$ such that $X''$ has cardinality $m'$.
            %Then $X'$ has cardinality $n'$.
            %Let $f'(z)=f(z)$ for $z \in X'$.
            %$\dom{f'} = X'$.
            %$\ran{f'} = Y'$.
            %$f'$ is a bijection from $X'$ to $Y'$.
            %Let $g'(z) = g(z)$ for $z \in X'$.
            %Then $g'$ is injective.
            %Then $g'$ is a bijection from $X'$ to $Y'$ by \cref{rels,id_elem_rels,times_empty_right,powerset_emptyset,double_complement_union,unions_cons,union_eq_cons,union_as_unions,unions_pow,cons_absorb,setminus_self,bijections_dom,ran_converse,id_apply,apply,unions_emptyset,img_emptyset,zero_is_empty}.
            %Define $G : X \to Y$ such that $G(z)=
            %\begin{cases}
            %    g'(z) & \text{if} z \in X' \\
            %    y' & \text{if} z = x'
            %\end{cases}$
            %$G = g$.
            %Follows by \cref{double_relative_complement,fun_to_surj,bijections,funs_surj_iff,bijections_to_funs,neq_witness,surj,funs_elim,setminus_self,cons_subseteq_iff,cardinality,ordinal_empty_or_emptyset_elem,naturals_inductive_set,natural_number_is_ordinal_for_all,foundation,inter_eq_left_implies_subseteq,inter_emptyset,cons_subseteq_intro,emptyset_subseteq}.
            Omitted.
    \end{byCase}
    %$\converse{f}$ is a bijection from $Y$ to $X$.
    %Let $h = g \circ \converse{f}$.
    %It suffices to show that $\ran{g} = Y$ by \cref{fun_to_surj,dom_converse,bijections}.
    %It suffices to show that for all $y \in Y$ we have there exist $x \in X$ such that $g(x)=y$ by \cref{funs_ran,subseteq_antisymmetric,fun_ran_iff,apply,funs_elim,ran_converse,subseteq}.
%
    %Fix $y \in Y$.
    %Take $x \in X$ such that $\apply{\converse{f}}{y} = x$.

\end{proof}

\begin{lemma}\label{injective_functions_is_bijection_if_bijection_there_is_other_bijection}
    Suppose $f$ is a bijection from $X$ to $Y$.
    Suppose $g$ is a function from $X$ to $Y$.
    Suppose $g$ is injective.
    Suppose $Y$ is finite. 
    Then $g$ is a bijection from $X$ to $Y$.
\end{lemma}
\begin{proof}
    There exist $n \in \naturals$ such that $Y$ has cardinality $n$ by \cref{cardinality,injective_functions_is_bijection_if_bijection_there_is_other_bijection_1,finite}.
    Follows by \cref{injective_functions_is_bijection_if_bijection_there_is_other_bijection_1,cardinality,equinum_tran,equinum_sym,equinum,finite}.
\end{proof}



\begin{lemma}\label{naturals_bijection_implies_eq}
    Suppose $n,m \in \naturals$.
    Suppose $f$ is a bijection from $n$ to $m$.
    Then $n = m$.
\end{lemma}
\begin{proof}
    $n$ is finite.
    $m$ is finite.
    Suppose not.
    Then $n < m$ or $m < n$.
    \begin{byCase}
        \caseOf{$n < m$.}
            Then $n \in m$.
            There exist $x \in m$ such that $x \notin n$.
            $\identity{n}$ is a function from $n$ to $m$.
            $\identity{n}$ is injective.
            $\apply{\identity{n}}{n} = n$ by \cref{id_ran,ran_of_surj,bijections,injective_functions_is_bijection_if_bijection_there_is_other_bijection}.
            Follows by \cref{inhabited,regularity,function_apply_default,apply,id_dom,in_irrefl,function_member_elim,bijections_dom,zero_is_empty,bijection_is_function,foundation,bijections,ran_of_surj,dom_converse,converse_emptyset,dom_emptyset}.
        \caseOf{$m < n$.}
            Then $m \in n$.
            There exist $x \in n$ such that $x \notin m$.
            $\converse{f}$ is a bijection from $m$ to $n$.
            $\identity{m}$ is a function from $m$ to $n$.
            $\identity{m}$ is injective.
            $\apply{\identity{m}}{m} = m$ by \cref{id_ran,ran_of_surj,bijections,injective_functions_is_bijection_if_bijection_there_is_other_bijection}.
            Follows by \cref{inhabited,regularity,function_apply_default,apply,id_dom,in_irrefl,function_member_elim,bijections_dom,zero_is_empty,bijection_is_function,foundation,bijections,ran_of_surj,dom_converse,converse_emptyset,dom_emptyset}.
    \end{byCase}
\end{proof}

\begin{lemma}\label{naturals_eq_iff_bijection}
    Suppose $n,m \in \naturals$.
    $n = m$ iff there exist $f$ such that $f$ is a bijection from $n$ to $m$.
\end{lemma}
\begin{proof}
    We show that if $n = m$ then there exist $f$ such that $f$ is a bijection from $n$ to $m$.
    \begin{subproof}
        Trivial.
    \end{subproof}
    We show that for all $k \in \naturals$ we have if there exist $f$ such that $f$ is a bijection from $k$ to $m$ then $k = m$.
    \begin{subproof}%[Proof by \in-induction on $k$]
        %Assume $k \in \naturals$.
        %\begin{byCase}
        %    \caseOf{$k = \zero$.} 
        %        Trivial.
        %    \caseOf{$k \neq \zero$.}
        %        \begin{byCase}
        %            \caseOf{$m = \zero$.}
        %                Trivial.
        %            \caseOf{$m \neq \zero$.}
        %                Take $k' \in \naturals$ such that $\suc{k'} = k$.
        %                Then $k' \in k$.
        %                Take $m' \in \naturals$ such that $m = \suc{m'}$.
        %                Then $m' \in m$.
        %                
        %        \end{byCase}
        %\end{byCase}
    \end{subproof}
\end{proof}

\begin{lemma}\label{seq_from_zero_suc_cardinality_eq_upper_border}
    Suppose $n,m \in \naturals$.
    Suppose $\seq{\zero}{n}$ has cardinality $\suc{m}$.
    Then $n = m$.
\end{lemma}
\begin{proof}
    We have $\seq{\zero}{n} = \suc{n}$.
    Take $f$ such that $f$ is a bijection from $\seq{\zero}{n}$ to $\suc{m}$.
    Therefore $n=m$ by \cref{suc_injective,naturals_inductive_set,cardinality,naturals_eq_iff_bijection}.
\end{proof}

\begin{lemma}\label{seq_from_zero_cardinality_eq_upper_border_set_eq}
    Suppose $n,m \in \naturals$.
    Suppose $\seq{\zero}{n}$ has cardinality $\suc{m}$.
    Then $\seq{\zero}{n} = \seq{\zero}{m}$.
\end{lemma}

\begin{proposition}\label{existence_normal_ordered_urysohn}
    Let $X$ be a urysohn space.
    Suppose $U$ is a urysohnchain of $X$.
    Suppose $\dom{U}$ is finite.
    Suppose $U$ is inhabited.
    Then there exist $V,f$ such that $V$ is a urysohnchain of $X$ and $f$ is consistent on $V$ to $U$ and $V$ is normal ordered.
\end{proposition}
\begin{proof}
    Take $n$ such that $\dom{U}$ has cardinality $n$ by \cref{ran_converse,cardinality,finite}.
    \begin{byCase}
        \caseOf{$n = \zero$.} 
            Omitted.
        \caseOf{$n \neq \zero$.}
            Take $k$ such that $k \in \naturals$ and $\suc{k}=n$. 
            We have $\dom{U} \subseteq \naturals$.
            $\dom{U}$ is inhabited by \cref{downward_closure,downward_closure_iff,rightunique,function_member_elim,inhabited,chain_of_subsets,urysohnchain,sequence}.
            $\dom{U}$ has cardinality $\suc{k}$.
            We show that there exist $F$ such that $F$ is a bijection from $\seq{\zero}{k}$ to $\dom{U}$ and for all $n',m' \in \seq{\zero}{k}$ such that $n' < m'$ we have $F(n') < F(m')$.
            \begin{subproof}
                For all $M \subseteq \naturals$ such that $M$ is inhabited we have there exist $f,k$ such that $f$ is a bijection from $\seq{\zero}{k}$ to $M$ and $M$ has cardinality $\suc{k}$ and for all $n,m \in \seq{\zero}{k}$ such that $n < m$ we have $f(n) < f(m)$.
                We have $\dom{U} \subseteq \naturals$.
                $\dom{U}$ is inhabited.
                Therefore there exist $f$ such that there exist $k'$ such that $f$ is a bijection from $\seq{\zero}{k'}$ to $\dom{U}$ and $\dom{U}$ has cardinality $\suc{k'}$ and for all $n',m' \in \seq{\zero}{k'}$ such that $n' < m'$ we have $f(n') < f(m')$.
                Take $f$ such that there exist $k'$ such that $f$ is a bijection from $\seq{\zero}{k'}$ to $\dom{U}$ and $\dom{U}$ has cardinality $\suc{k'}$ and for all $n',m' \in \seq{\zero}{k'}$ such that $n' < m'$ we have $f(n') < f(m')$.
                Take $k'$ such that $f$ is a bijection from $\seq{\zero}{k'}$ to $\dom{U}$ and $\dom{U}$ has cardinality $\suc{k'}$ and for all $n',m' \in \seq{\zero}{k'}$ such that $n' < m'$ we have $f(n') < f(m')$.
                $\seq{\zero}{k'}$ has cardinality $\suc{k}$ by \cref{omega_is_an_ordinal,suc,ordinal_transitivity,bijection_converse_is_bijection,seq_zero_to_n_eq_to_suc_n,cardinality,bijections_dom,bijection_circ}.
                $\seq{\zero}{k'} = \seq{\zero}{k}$ by \cref{omega_is_an_ordinal,seq_from_zero_cardinality_eq_upper_border_set_eq,suc_subseteq_implies_in,suc_subseteq_elim,ordinal_suc_subseteq,cardinality}.
                %We show that $\seq{\zero}{k'} = \seq{\zero}{k}$.
                %\begin{subproof}
                %    We show that $\seq{\zero}{k'} \subseteq \seq{\zero}{k}$.
                %    \begin{subproof}
                %        It suffices to show that for all $y \in \seq{\zero}{k'}$ we have $y \in \seq{\zero}{k}$.
                %        Fix $y \in \seq{\emptyset}{k'}$.
                %        Then $y \leq k'$.
                %        Therefore $y \in k'$ or $y = k'$ by \cref{omega_is_an_ordinal,suc_intro_self,ordinal_transitivity,cardinality,rless_eq_in_for_naturals,m_to_n_set}.
                %        
                %        Therefore $y \in \suc{k}$.
                %        Therefore $y \in \seq{\emptyset}{k}$.
                %    \end{subproof}
                %    We show that for all $y \in \seq{\zero}{k}$ we have $y \in \seq{\zero}{k'}$.
                %    \begin{subproof}
                %        Fix $y \in \seq{\emptyset}{k}$.
                %    \end{subproof}
                %\end{subproof}
            \end{subproof}
            Take $F$ such that $F$ is a bijection from $\seq{\zero}{k}$ to $\dom{U}$ and for all $n',m' \in \seq{\zero}{k}$ such that $n' < m'$ we have $F(n') < F(m')$.
            Let $N = \seq{\zero}{k}$.
            Let $M = \pow{X}$.
            Define $V : N \to M$ such that $V(n)=
            \begin{cases}
                \at{U}{F(n)} & \text{if} n \in N
            \end{cases}$
            $\dom{V} = \seq{\zero}{k}$.
            We show that $V$ is a urysohnchain of $X$.
            \begin{subproof}
                It suffices to show that $V$ is a chain of subsets in $X$ and for all $n,m \in \dom{V}$ such that $n < m$ we have $\closure{\at{V}{n}}{X} \subseteq \interior{\at{V}{m}}{X}$.
                We show that $V$ is a chain of subsets in $X$.
                \begin{subproof}
                    It suffices to show that $V$ is a sequence and for all $n \in \dom{V}$ we have $\at{V}{n} \subseteq \carrier[X]$ and for all $m \in \dom{V}$ such that $m > n$ we have $\at{V}{n} \subseteq \at{V}{m}$.
                    $V$ is a sequence by \cref{m_to_n_set,sequence,subseteq}.
                    It suffices to show that for all $n \in \dom{V}$ we have $\at{V}{n} \subseteq \carrier[X]$ and for all $m \in \dom{V}$ such that $m > n$ we have $\at{V}{n} \subseteq \at{V}{m}$.
                    Fix $n \in \dom{V}$.
                    Then $\at{V}{n} \subseteq \carrier[X]$ by \cref{ran_converse,seq_zero_to_n_eq_to_suc_n,in_irrefl}.
                    It suffices to show that for all $m$ such that $m \in \dom{V}$ and $n \rless m$ we have $\at{V}{n} \subseteq \at{V}{m}$.
                    Fix $m$ such that $m \in \dom{V}$ and $n \rless m$.
                    Follows by \cref{ran_converse,seq_zero_to_n_eq_to_suc_n,in_irrefl}.
                \end{subproof}
                It suffices to show that for all $n \in \dom{V}$ we have for all $m$ such that $m \in \dom{V} \land n \rless m$ we have $\closure{\at{V}{n}}{X} \subseteq \interior{\at{V}{m}}{X}$.
                Fix $n \in \dom{V}$.
                Fix $m$ such that $m \in \dom{V} \land n \rless m$.
                Follows by \cref{ran_converse,seq_zero_to_n_eq_to_suc_n,in_irrefl}.
            \end{subproof}
            We show that $F$ is consistent on $V$ to $U$.
            \begin{subproof}
                It suffices to show that $F$ is a bijection from $\dom{V}$ to $\dom{U}$ and for all $n,m \in \dom{V}$ such that $n < m$ we have $F(n) < F(m)$ by \cref{bijection_of_urysohnchains}.
                $F$ is a bijection from $\dom{V}$ to $\dom{U}$.
                It suffices to show that for all $n \in \dom{V}$ we have for all $m$ such that $m \in \dom{V}$ and $n \rless m$ we have $f(n) < f(m)$.
                Fix $n \in \dom{V}$.
                Fix $m$ such that $m \in \dom{V}$ and $n \rless m$.
                Follows by \cref{ran_converse,seq_zero_to_n_eq_to_suc_n,in_irrefl}.
            \end{subproof}
            $V$ is normal ordered.
    \end{byCase}
    
\end{proof}


\begin{definition}\label{staircase}
    $f$ is a staircase function adapted to $U$ in $X$ iff $U$ is a urysohnchain of $X$ and for all $x,n,m,k$ such that $k = \max{\dom{U}}$ and $n,m \in \dom{U}$ and $n$ follows $m$ in $\dom{U}$ and $x \in (\at{U}{m} \setminus \at{U}{n})$ we have $f(x)= \rfrac{m}{k}$.
\end{definition}

\begin{definition}\label{staircase_sequence}
    $f$ is staircase sequence of $U$ iff $f$ is a sequence and $U$ is a lifted urysohnchain of $X$ and $\dom{U} = \dom{f}$ and for all $n \in \dom{U}$ we have $\at{f}{n}$ is a staircase function adapted to $\at{U}{n}$ in $U$.
\end{definition}

\begin{definition}
    
\end{definition}



\begin{theorem}\label{urysohnsetinbeetween}
    Let $X$ be a urysohn space.
    Suppose $A,B \in \closeds{X}$.
    Suppose $\closure{A}{X} \subseteq \interior{B}{X}$.
    Suppose $\carrier[X]$ is inhabited.
    There exist $U \subseteq \carrier[X]$ such that $U$ is closed in $X$ and $\closure{A}{X} \subseteq \interior{U}{X} \subseteq \closure{U}{X} \subseteq \interior{B}{X}$.
\end{theorem}
\begin{proof}
    Omitted.
\end{proof}


\begin{theorem}\label{induction_on_urysohnchains}
    Let $X$ be a urysohn space.
    Suppose $U_0$ is a sequence.
    Suppose $U_0$ is a chain of subsets in $X$.
    Suppose $U_0$ is a urysohnchain of $X$.
    There exist $U$ such that $U$ is a sequence and $\dom{U} = \naturals$ and $\at{U}{\zero} = U_0$ and for all $n \in \dom{U}$ we have $\at{U}{n}$ is a urysohnchain of $X$ and $\at{U}{\suc{n}}$ is a minimal finer extention of $\at{U}{n}$ in $X$.
\end{theorem}
\begin{proof}
    %$U_0$ is a urysohnchain of $X$.
    %It suffices to show that for all $V$ such that $V$ is a urysohnchain of $X$ there exist $V'$ such that $V'$ is a urysohnchain of $X$ and $V'$ is a minimal finer extention of $V$ in $X$.
    Omitted.
\end{proof}





\begin{theorem}\label{urysohn}
    Let $X$ be a urysohn space.
    Suppose $A,B \in \closeds{X}$.
    Suppose $A \inter B$ is empty.
    Suppose $\carrier[X]$ is inhabited.
    There exist $f$ such that $f \in \funs{\carrier[X]}{\intervalclosed{\zero}{1}}$ 
    and $f(A) = \zero$ and $f(B)= 1$ and $f$ is continuous.
\end{theorem}
\begin{proof}
    Let $X' = \carrier[X]$.
    Let $N = \{\zero, 1\}$.
    $1 = \suc{\zero}$.
    $1 \in \naturals$ and $\zero \in \naturals$.
    $N \subseteq \naturals$.
    Let $A' = (X' \setminus B)$.
    $B \subseteq X'$ by \cref{powerset_elim,closeds}.
    $A \subseteq X'$.
    Therefore $A \subseteq A'$.
    Define $U_0: N \to \{A, A'\}$ such that $U_0(n) =
    \begin{cases}
        A  &\text{if} n = \zero \\
        A' &\text{if} n = 1
    \end{cases}$
    $U_0$ is a function.
    $\dom{U_0} = N$.
    $\dom{U_0} \subseteq \naturals$ by \cref{ran_converse}. 
    $U_0$ is a sequence.
    We have $1, \zero \in N$.
    We show that $U_0$ is a chain of subsets in $X$.
    \begin{subproof}
        We have $\dom{U_0} \subseteq \naturals$.
        We have for all $n \in \dom{U_0}$ we have $\at{U_0}{n} \subseteq \carrier[X]$ by \cref{topological_prebasis_iff_covering_family,union_as_unions,union_absorb_subseteq_left,subset_transitive,setminus_subseteq}.
        We have $\dom{U_0} = \{\zero, 1\}$.

        It suffices to show that for all $n \in \dom{U_0}$ we have for all $m \in \dom{U_0}$ such that $m > n$ we have $\at{U_0}{n} \subseteq \at{U_0}{m}$.

        Fix $n \in \dom{U_0}$.
        Fix $m \in \dom{U_0}$.

        \begin{byCase}
            \caseOf{$n \neq \zero$.} 
                Trivial.
            \caseOf{$n = \zero$.} 
                \begin{byCase}
                    \caseOf{$m = \zero$.} 
                        Trivial.
                    \caseOf{$m \neq \zero$.}
                        We have $A \subseteq A'$.
                        We have $\at{U_0}{\zero} = A$ by assumption.
                        We have $\at{U_0}{1}= A'$ by assumption.
                        Follows by \cref{powerset_elim,emptyset_subseteq,union_as_unions,union_absorb_subseteq_left,subseteq_pow_unions,ran_converse,subseteq,subseteq_antisymmetric,suc_subseteq_intro,apply,powerset_emptyset,emptyset_is_ordinal,notin_emptyset,ordinal_elem_connex,omega_is_an_ordinal,prec_is_ordinal}.
                \end{byCase}
        \end{byCase}
    \end{subproof}

    We show that $U_0$ is a urysohnchain of $X$.
    \begin{subproof}
        It suffices to show that for all $n \in \dom{U_0}$ we have for all $m \in \dom{U_0}$ such that $n < m$ we have $\closure{\at{U_0}{n}}{X} \subseteq \interior{\at{U_0}{m}}{X}$.
        Fix $n \in \dom{U_0}$.
        Fix $m \in \dom{U_0}$.
        \begin{byCase}
            \caseOf{$n \neq \zero$.} 
                Follows by \cref{ran_converse,upair_iff,one_in_reals,order_reals_lemma0,reals_axiom_zero_in_reals,reals_one_bigger_zero,reals_order}.
            \caseOf{$n = \zero$.} 
                \begin{byCase}
                    \caseOf{$m = \zero$.} 
                        Trivial.
                    \caseOf{$m \neq \zero$.}
                        Follows by \cref{setminus_emptyset,setdifference_eq_intersection_with_complement,setminus_self,interior_carrier,complement_interior_eq_closure_complement,subseteq_refl,closure_interior_frontier_is_in_carrier,emptyset_subseteq,closure_is_minimal_closed_set,inter_lower_right,inter_lower_left,subseteq_transitive,interior_of_open,is_closed_in,closeds,union_absorb_subseteq_right,ordinal_suc_subseteq,ordinal_empty_or_emptyset_elem,union_absorb_subseteq_left,union_emptyset,topological_prebasis_iff_covering_family,inhabited,notin_emptyset,subseteq,union_as_unions,natural_number_is_ordinal}.
                \end{byCase}
        \end{byCase}
    \end{subproof}
    %We are done with the first step, now we want to prove that we have U a sequence of these chain with U_0 the first chain.

    We show that there exist $U$ such that $U$ is a sequence and $\dom{U} = \naturals$ and $\at{U}{\zero} = U_0$ and for all $n \in \dom{U}$ we have $\at{U}{n}$ is a urysohnchain of $X$ and $\at{U}{\suc{n}}$ is a minimal finer extention of $\at{U}{n}$ in $X$.
    \begin{subproof}
        Follows by \cref{chain_of_subsets,urysohnchain,induction_on_urysohnchains}.
    \end{subproof}
    Take $U$ such that $U$ is a lifted urysohnchain of $X$ and $\at{U}{\zero} = U_0$.

    We show that there exist $S$ such that $S$ is staircase sequence of $U$.
    \begin{subproof}
        Omitted.
    \end{subproof}
    Take $S$ such that $S$ is staircase sequence of $U$.

    For all $x \in \carrier[X]$ we have there exist $r,R$ such that $r \in \reals$ and $R$ is a sequence of reals and $\dom{R} = \naturals$ and $R$ converge to $r$ and for all $n \in \naturals$ we have $\at{R}{n} = \apply{\at{S}{n}}{x}$.

    We show that for all $x \in \carrier[X]$ there exists $r \in \intervalclosed{a}{b}$ such that for .


    
\end{proof}

\begin{theorem}\label{safe}
    Contradiction.     
\end{theorem}





%
%Ideen:
%Eine folge ist ein Funktion mit domain \subseteq Natürlichenzahlen. als predicat
%
%zulässig und verfeinerung von ketten als predicat definieren. 
%
%limits und punkt konvergenz als prädikat.
%
%
%Vor dem Beweis vor dem eigentlichen Beweis.
%die abgeleiteten Funktionen
%
%\derivedstiarcasefunction on A
%
%abbreviation: \at{f}{n} = f_{n}
%
%
%TODO:
%Reals ist ein topologischer Raum
%

