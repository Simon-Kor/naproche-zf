\import{topology/topological-space.tex}
\import{topology/separation.tex}
\import{topology/continuous.tex}
\import{topology/basis.tex}
\import{numbers.tex}
\import{function.tex}
\import{set.tex}
\import{cardinal.tex}
\import{relation.tex}
\import{relation/uniqueness.tex}
\import{set/cons.tex}
\import{set/powerset.tex}
\import{set/fixpoint.tex}
\import{set/product.tex}

\section{Urysohns Lemma}



\begin{abbreviation}\label{urysohnspace}
    $X$ is a urysohn space iff
    $X$ is a topological space and
    for all $A,B \in \closeds{X}$ such that $A \inter B = \emptyset$
    we have there exist $A',B' \in \opens[X]$
    such that  $A \subseteq A'$ and $B \subseteq B'$ and $A' \inter B' = \emptyset$.    
\end{abbreviation}


\begin{definition}\label{intervalclosed}
    $\intervalclosed{a}{b} = \{x \in \reals \mid a \leq x \leq b\}$.
\end{definition}




\begin{theorem}\label{urysohn}
    Let $X$ be a urysohn space.
    Suppose $A,B \in \closeds{X}$.
    Suppose $A \inter B$ is empty.
    Suppose $\carrier[X]$ is inhabited.
    There exist $f$ such that $f \in \funs{\carrier[X]}{\intervalclosed{\zero}{1}}$ 
    and $f(A) = \zero$ and $f(B)= 1$ and $f$ is continuous.
\end{theorem}
\begin{proof}


    Define $f : X \to \reals$ such that  $f(x) = $
    \begin{cases}
        &(x + x) , x \in X
        % & x ,x \in X    <- will result in technicly ambigus parse
    \end{cases}

    Trivial.
    
    
\end{proof}

\begin{theorem}\label{safe}
    Contradiction.     
\end{theorem}
