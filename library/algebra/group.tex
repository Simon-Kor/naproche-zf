\import{algebra/monoid.tex}
\section{Group}\label{form_sec_group}

\begin{struct}\label{group}
    A group $G$ is a monoid such that
    \begin{enumerate}
        \item\label{group_inverse} for all $g \in \carrier[G]$ there exist $h \in \carrier[G]$ such that $\mul[G](g, h) =\neutral[G]$.
    \end{enumerate} 
\end{struct}   

\begin{corollary}\label{group_implies_monoid}
    Let $G$ be a group. Then $G$ is a monoid.
\end{corollary}

\begin{abbreviation}\label{cfourdot}
    $g \cdot h = \mul(g,h)$.
\end{abbreviation}

\begin{lemma}\label{neutral_is_idempotent}
    Let $G$ be a group. $\neutral[G]$ is a idempotent element of $G$.
\end{lemma}

\begin{lemma}\label{group_divison_right}
    Let $G$ be a group. Let $a,b,c \in G$.
    Then $a \cdot c = b \cdot c$ iff $a = b$.
\end{lemma}
\begin{proof}
    Take $a,b,c \in G$ such that $a \cdot c = b \cdot c$.
    There exist $c' \in G$ such that $c \cdot c' = \neutral[G]$.
    Therefore $a \cdot c = b \cdot c$ iff $(a \cdot c) \cdot c' = (b \cdot c) \cdot c'$.
    \begin{align*}
        (a \cdot c) \cdot c'
            &= a \cdot (c \cdot c')
                \explanation{by \cref{semigroup_assoc,group_implies_monoid,monoid_implies_semigroup}}\\
            &= a \cdot \neutral[G]
                \explanation{by \cref{group_inverse}}\\
            &= a
                \explanation{by \cref{group_implies_monoid,monoid_right}}
    \end{align*}
    \begin{align*}
        (b \cdot c) \cdot c'
            &= b \cdot (c \cdot c')
                \explanation{by \cref{semigroup_assoc,group_implies_monoid,monoid_implies_semigroup}}\\
            &= b \cdot \neutral[G]
                \explanation{by \cref{group_inverse}}\\
            &= b
                \explanation{by \cref{group_implies_monoid,monoid_right}}
    \end{align*}
    $(a \cdot c) \cdot c' = (b \cdot c) \cdot c'$ iff $a \cdot c = b \cdot c$ by assumption. 
    $a = b$ iff $a \cdot c = b \cdot c$ by assumption.
\end{proof}


\begin{proposition}\label{leftinverse_eq_rightinverse}
    Let $G$ be a group and assume $a \in G$.
    Then there exist $b\in G$ 
    such that $a \cdot b = \neutral[G]$ and $b \cdot a = \neutral[G]$.
\end{proposition}
\begin{proof}
    There exist $b \in G$ such that $a \cdot b = \neutral[G]$.
    There exist $c \in G$ such that $b \cdot c = \neutral[G]$.
    $a \cdot b = \neutral[G]$.
    $(a \cdot b) \cdot c = (\neutral[G]) \cdot c$.
    $(a \cdot b) \cdot c = a \cdot (b \cdot c)$.
    $a \cdot \neutral[G] = \neutral[G] \cdot c$.
    $c = c \cdot \neutral[G]$.
    $c = \neutral[G] \cdot c$.
    $a \cdot \neutral[G] = c \cdot \neutral[G]$.
    $a \cdot \neutral[G] = c$ by \cref{monoid_right,group_divison_right}.
    $a = c$ by \cref{monoid_right,group_divison_right,neutral_is_idempotent}.
    $b \cdot a = b \cdot c$.
    $b \cdot a = \neutral[G]$.
\end{proof}

\begin{definition}\label{group_abelian}
    $G$ is an abelian group iff $G$ is a group and for all $g,h \in G$ $\mul[G](g,h) = \mul[G](h,g)$. 
\end{definition}


\begin{definition}\label{group_automorphism}
    Let $f$ be a function. $f$ is a group-automorphism iff $G$ is a group and $\dom{f}=G$ and $\ran{f}=G$. 
\end{definition}

\begin{definition}\label{trivial_group}
    $G$ is the trivial group iff $G$ is a group and $\{\neutral[G]\}=G$.
\end{definition}

\begin{theorem}\label{trivial_implies_abelian}
    Let $G$ be a group.
    Suppose $G$ is the trivial group.
    Then $G$ is an abelian group.
\end{theorem}
