\import{order/order.tex}
\import{relation.tex}

\section{The real numbers}

\begin{signature}
    $\reals$ is a set.
\end{signature}

\begin{signature}
    $x + y$ is a set.
\end{signature}

\begin{signature}
    $x \rmul y$ is a set.
\end{signature}

\subsection{Basic axioms of the reals}

\begin{axiom}\label{reals_axiom_zero_in_reals}
    $\zero \in \reals$.
\end{axiom}

\begin{axiom}\label{one_in_reals}
    $1 \in \reals$.
\end{axiom}

\begin{axiom}\label{zero_neq_one}
    $\zero \neq 1$.
\end{axiom}

\begin{definition}\label{reals_definition_order_def}
    $x < y$ iff there exist $z \in \reals$ such that $x + (z \rmul z) = y$.
\end{definition}

%\begin{axiom}\label{reals_axiom_order}
%    $\lt[\reals]$ is an order on $\reals$.
%\end{axiom}

%\begin{abbreviation}\label{lesseq_on_reals}
%    $x \leq y$ iff $(x,y) \in \lt[\reals]$.
%\end{abbreviation}

\begin{abbreviation}\label{less_on_reals}
    $x \leq y$ iff it is wrong that $y < x$.
\end{abbreviation}

\begin{abbreviation}\label{greater_on_reals}
    $x > y$ iff $y \leq x$.
\end{abbreviation}

\begin{abbreviation}\label{greatereq_on_reals}
    $x \geq y$ iff it is wrong that $x < y$.
\end{abbreviation}

\begin{axiom}\label{reals_axiom_dense}
    For all $x,y \in \reals$ if $x < y$ then 
    there exist $z \in \reals$ such that $x < z$ and $z < y$.
\end{axiom}

\begin{axiom}\label{reals_axiom_assoc}
    For all $x,y,z \in \reals$ $(x + y) + z = x + (y + z)$ and $(x \rmul y) \rmul z = x \rmul (y \rmul z)$.
\end{axiom}

\begin{axiom}\label{reals_axiom_kommu}
    For all $x,y \in \reals$ $x + y = y + x$ and $x \rmul y = y \rmul x$.
\end{axiom}
  
\begin{axiom}\label{reals_axiom_zero}
    For all $x \in \reals$ $x + \zero = x$. 
\end{axiom}

\begin{axiom}\label{reals_axiom_one}
    For all $x \in \reals$ we have $x \rmul 1 = x$.
\end{axiom}

\begin{axiom}\label{reals_axiom_add_invers}
    For all $x \in \reals$ there exist $y \in \reals$ such that $x + y = \zero$.
\end{axiom}

\begin{axiom}\label{reals_axiom_mul_invers}
    For all $x \in \reals$ such that $x \neq \zero$ there exist $y \in \reals$ such that $x \rmul y = 1$.
\end{axiom}

\begin{axiom}\label{reals_axiom_disstro1}
    For all $x,y,z \in \reals$ $x \rmul (y + z) = (x \rmul y) + (x \rmul z)$.
\end{axiom}

\begin{axiom}\label{reals_axiom_dedekind_complete}
    For all $X,Y,x,y$ such that $X,Y \subseteq \reals$ and $x \in X$ and $y \in Y$ and $x < y$ we have there exist $z \in \reals$
    such that $x < z < y$.
\end{axiom}

\begin{proposition}\label{reals_disstro2}
    For all $x,y,z \in \reals$ $(y + z) \rmul x = (y \rmul x) + (z \rmul x)$.
\end{proposition}
\begin{proof}
    Omitted.
\end{proof}

\begin{proposition}\label{reals_reducion_on_addition}
    For all $x,y,z \in \reals$ if $x + y = x + z$ then $y = z$.
\end{proposition}



%\begin{signature}\label{invers_is_set}
%    $\addInv{y}$ is a set.
%\end{signature}

%\begin{definition}\label{add_inverse}
%    $\addInv{y} = \{ x \mid \exists k \in \reals. k + y = \zero \land x \in k\}$.
%\end{definition}
    

%\begin{definition}\label{add_inverse_natural_language}
%    $x$ is additiv inverse of $y$ iff $x = \addInv{y}$. 
%\end{definition}

%\begin{lemma}\label{rminus}
%    $x \rminus \addInv{x} = \zero$.
%\end{lemma}

\begin{abbreviation}\label{is_positiv}
    $x$ is positiv iff $x > \zero$.
\end{abbreviation}

\begin{lemma}\label{reals_add_of_positiv}
    Let $x,y \in \reals$.
    Suppose $x$ is positiv and $y$ is positiv.
    Then $x + y$ is positiv.
\end{lemma}
\begin{proof}
    Omitted.
\end{proof}

\begin{lemma}\label{reals_mul_of_positiv}
    Let $x,y \in \reals$.
    Suppose $x$ is positiv and $y$ is positiv.
    Then $x \rmul y$ is positiv.
\end{lemma}
\begin{proof}
    Omitted.
\end{proof}



\begin{lemma}\label{order_reals_lemma0}
    For all $x \in \reals$ we have not $x < x$.
\end{lemma}
\begin{proof}
    Omitted.
\end{proof}


\begin{lemma}\label{order_reals_lemma1}
    Let $x,y,z \in \reals$.
    Suppose $\zero < x$. 
    Suppose $y < z$. 
    Then $(y \rmul x) < (z \rmul x)$.
\end{lemma}
\begin{proof}
    Omitted.
    %There exist $k \in \reals$ such that $y + k = z$ and $k > \zero$ by \cref{reals_definition_order_def}.
    %\begin{align*}
    %    (z \rmul x) \\
    %    &= ((y + k) \rmul x) \\
    %    &= ((y \rmul x) + (k \rmul x)) \explanation{by \cref{reals_disstro2}}
    %\end{align*} 
    %Then $(k \rmul x) > \zero$. 
    %Therefore $(z \rmul x) > (y \rmul x)$.
\end{proof}

\begin{lemma}\label{order_reals_lemma2}
    Let $x,y,z \in \reals$.
    Suppose $\zero < x$. 
    Suppose $y < z$. 
    Then $(x \rmul y) < (x \rmul z)$.
\end{lemma}
\begin{proof}
    Omitted.
\end{proof}


\begin{lemma}\label{order_reals_lemma3}
    Let $x,y,z \in \reals$.
    Suppose $\zero < x$. 
    Suppose $y < z$. 
    Then $(x \rmul z) < (x \rmul y)$.
\end{lemma}
\begin{proof}
    Omitted.
\end{proof}

\begin{lemma}\label{order_reals_lemma00}
    For all $x,y \in \reals$ such that $x > y$ we have $x \geq y$.
\end{lemma}


\begin{lemma}\label{order_reals_lemma5}
    For all $x,y \in \reals$ such that $x < y$ we have $x \leq y$.
\end{lemma}
\begin{proof}
    Omitted.
\end{proof}

\begin{lemma}\label{order_reals_lemma6}
    For all $x,y \in \reals$ such that $x \leq y \leq x$ we have $x=y$.
\end{lemma}
\begin{proof}
    Omitted.
\end{proof}

\begin{lemma}\label{reals_minus}
    Assume $x,y \in \reals$. If $x \rminus y = \zero$ then $x=y$.
\end{lemma}
\begin{proof}
    Omitted.
\end{proof}

\begin{definition}\label{upper_bound}
    $x$ is an upper bound of $X$ iff for all $y \in X$ we have $x > y$.
\end{definition}

\begin{definition}\label{least_upper_bound}
    $x$ is a least upper bound of $X$ iff $x$ is an upper bound of $X$ and for all $y$ such that $y$ is an upper bound of $X$ we have $x \leq y$.
\end{definition}

\begin{lemma}\label{supremum_unique}
    %Let $x,y \in \reals$ and let $X$ be a subset of $\reals$.
    If $x$ is a least upper bound of $X$ and $y$ is a least upper bound of $X$ then $x = y$.
\end{lemma}
\begin{proof}
    Omitted.
\end{proof}

\begin{definition}\label{supremum_reals}
    $x$ is the supremum of $X$ iff $x$ is a least upper bound of $X$.
\end{definition}




\begin{definition}\label{lower_bound}
    $x$ is an lower bound of $X$ iff for all $y \in X$ we have $x < y$.
\end{definition}

\begin{definition}\label{greatest_lower_bound}
    $x$ is a greatest lower bound of $X$ iff $x$ is an lower bound of $X$ and for all $y$ such that $y$ is an lower bound of $X$ we have $x \geq y$.
\end{definition}

\begin{lemma}\label{infimum_unique}
    If $x$ is a greatest lower bound of $X$ and $y$ is a greatest lower bound of $X$ then $x = y$.
\end{lemma}
\begin{proof}
    Omitted.
\end{proof}

\begin{definition}\label{infimum_reals}
    $x$ is the supremum of $X$ iff $x$ is a greatest lower bound of $X$.
\end{definition}




\section{The natural numbers}


\begin{abbreviation}\label{def_suc}
    $\successor{n} = n + 1$.
\end{abbreviation}

\begin{inductive}\label{naturals_definition_as_subset_of_reals}
    Define $\nat \subseteq \reals$ inductively as follows.
    \begin{enumerate}
        \item $\zero \in \nat$.
        \item If $n\in \nat$, then $\successor{n} \in \nat$. 
    \end{enumerate}
\end{inductive}

\begin{lemma}\label{reals_order_suc}
    $n < \successor{n}$.
\end{lemma}

%\begin{proposition}\label{safe}
%    Contradiction.
%\end{proposition}

\section{The integers}

%\begin{definition}
%    $\integers = \{z \in \reals \mid z = \zero or \} $.
%\end{definition}